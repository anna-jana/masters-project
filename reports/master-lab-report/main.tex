\documentclass[13pt,a4paper,twoside,titlepage]{article}
\usepackage[ngerman, english]{babel}
\usepackage[utf8]{inputenc}
\usepackage{latexsym,amssymb,amsmath,amstext}
\usepackage{graphicx}
\usepackage{float}
\usepackage[backend=bibtex, sorting=none]{biblatex}
\usepackage{amsmath}
\usepackage{ dsfont }
\usepackage{csquotes}
\usepackage{slashed}
\usepackage{subfigure}
\usepackage{listings}
\usepackage{subfigure}
\usepackage{verbatim}
\usepackage{xcolor}
\usepackage{multicol}
\usepackage{feynmf}
\usepackage{abstract}
\usepackage{bbold}
\usepackage[a4paper, left=2cm, right=2cm, top=2cm, bottom=3cm, bindingoffset=10mm]{geometry}
\usepackage[compat=1.1.0]{tikz-feynman}
%\usepackage[onehalfspacing]{setspace}
\usepackage{tikz}
\usepackage{pgfplots}
\lstset{
  breaklines=true,
  postbreak=\mbox{\textcolor{red}{$\hookrightarrow$}\space},
}

\newcommand{\Tr}{\operatorname{Tr}}
\newcommand{\note}[1]{{\color{red}[\textbf{#1}]}}
\newcommand{\diff}{\ensuremath{\mathrm{d}}}

\bibliography{refs}

\begin{document}
%\pagenumbering{gobble}
\begin{titlepage}

\vspace*{3cm}

\centering
\textsc{\Large Masters Project - Lab Report}
% \textsc{\Large \bfseriedvanced Masters Project - Lab Report}
\\
\vspace*{1cm}
%\rule{\textwidth}{1pt}\\[0.1cm]
{\Large \bfseries
Spontaneous Baryogenesis using Axions
\\[0.1cm]}
%\rule{\textwidth}{1pt}
\vspace*{1cm}
{
\centering\Large
prepared by\\[0.2cm]
{\bfseries Jana Riess}\\[0.2cm]
}

\vspace*{2cm}

\begin{Large}
\begin{tabular}{ll}
%\bfseries Date of Submission: &21.01.2020\\
\end{tabular}
\end{Large}

\vspace*{1.5cm}

\end{titlepage}

\tableofcontents
\newpage

\newcommand{\jana}[1]{{\color{magenta}{#1}}}t


\section{Introduction}
\label{sec:introduction}

\noindent
\jana{Überarbeiten!!!!!}
In ????? Paul Dirac came up with an interpretation of negative energy states in his Dirac equations as anti-particles.
These anti-particles are a fundamental feature in theories of fundamental interactions and it was long believed that
nature is symmetric under the exchange of a particle with its anti-particle.
At high temperatures in the early universe particles and anti-particles have been present both in high number densities due to rapid pair-production and annihilations into two photons. After the universe was been cooled down below the freeze-out temperature of these processes, all annihilations into photons happend and only the asymmetry between particles and anti-particles remained. If
there were a symmetry between particles and anti-particles then only photons should remain.
But the universe we observe is made out of matter. Objects like galaxies made out of anti-matter have not been observed, hence some kind of speration between matter and anti-matter within our universe seems unlikely. ???????
Therefore some process has to have happend that created an asymmetry between particles and anti-particles. This process
is known as \emph{baryogenesis}.

\noindent
In this lab report we will discuss the issue of baryogenesis and one possible solution in detail, \emph{spontaneous baryogenesis} using \emph{axions}. Furthermore we show one example of how one can implement spontaneous baryogenesis during reheating using axions coupling via an SU(2) anomaly and lepton-number violating interactions from an effective Weinberg operator arising from a type-I see-saw mechanism.

\noindent
This report is organised as follows:
In section \ref{sec:baryogenesis} the general issue of baryogenesis is discussed and the most well known theories are presented briefly.
Then the lesser known approach of spontaneous baryogenesis is presented in section \ref{sec:spontaneour_bayrogenesis}.
For a concrete implementation of this idea, we cover the fundamentals of two prerequisites, namely lepton-number violation from the type-I see-saw mechanism i.e. right handed Majorana neutrinos in section \ref{sec:lapton_number_violations} and axions and their couplings as well as their cosmological dynamics in section \ref{sec:axions}. Then we will discuss the actual model in section \ref{sec:spont_bg_lepto_with_axions}.
Finally we will discuss potential extensions of this model in section \ref{sec:multi_fields} and conclude in section \ref{sec:conclusion_and_outlook} including an outlook on other possible models for baryogenesis using axions.

\section{Basics of Baryogenesis}
\label{sec:baryogenesis}

This section reviews the fundamentals of baryogenesis.

\subsection{Evolution of the Baryon Asymmetry}
First let us review the evolution of the baryon density. We know from collider experiments that at energies below $\sim 1 \, \mathrm{TeV}$ the baryon and lepton numbers are conserved.
Therefore the process of baryogenesis had to happen at higher energies.
At these energies the baryons/lepton and photons are in chemical equilibrium.
This means that from thermodynamics
\begin{align}
\label{eq:baryon_equilibrium}
n_b + n_{\bar{b}} \approx n_\gamma
\end{align}
At lower energies the baryons annihilate into photons. That is all except the baryon excess $n_B$. Now the baryon asymmetry $n_B$ is maximal. Those remaining baryons make up the baryon density today.

We want to quantify the amount of asymmetry between baryons and anti-baryons.
Today the asymmetry is maximal but
above temperatures of $\sim 1 \, \mathrm{GeV}$ baryons and anti-baryons are in chemical equilibrium.
Because of eq. \eqref{eq:baryon_equilibrium} it is convenient to express the asymmetry as
the ratio between the difference between the number densities of baryons $n_b$ and anti-baryons $n_{\hat{b}}$
over the number densities of photons \cite[eq. 3.52]{the_early_universe_kolb_and_turner}
\begin{align}
    n_\gamma = 2 \frac{\zeta(3)}{\pi^2} T^3
\end{align}
i.e. as the asymmetry parameter ?????
\begin{align}
    \eta_B = \frac{n_B}{n_\gamma}.
\end{align}
This quantity is conserved as the universe expands because both the baryon number density
and the photon density scale as $\sim 1/R^3 \sim T^3$.

However this is not conserved quantity if the co-moving number of photons changes due to annihilation
of particles e.g. baryons or leptons into photons.
Since the total co-moving entropy is conserved (the entropy of the annihilating particles is converted into the photons), the baryon abundance
\begin{align}
    \label{eq:baryon_abundance}
    Y_B := \frac{n_B}{s},
\end{align}
where
\begin{align}
    s = \frac{2 \pi^2}{45} g_{*, s}(T) T^3
\end{align}
with the effective relativistic degrees of freedom for the entropy density $g_{*, s}(T)$
is the entropy density
is conserved since both scale as $\sim T^3$.
Alternatively one can compute the present day baryon asymmetry $\eta_B^0$ from eq. \eqref{eq:baryon_abundance} by considering the ratio
\begin{align}
    \frac{\eta_B}{Y_B} = \frac{\pi^4}{45 \zeta(3)} g_{*, s}(T)
\end{align}
leading to
\begin{align}
    \eta_B^0 = \frac{g_{*,s}^0}{g_{*, s}^{\gg 1 \, \mathrm{Gev}}} \eta_B^{\gg 1 \, \mathrm{Gev}}.
\end{align}
where in the standard model one has $g_{*,s}^0 = 43/11$ and $g_{*, s}^{\gg 1 \, \mathrm{Gev}} = 427/4$ \cite[sec. 3.4]{the_early_universe_kolb_and_turner}.

The present value of the baryon asymmetry $\eta_B^0$ can be determined from separate measurements
of the baryon number density $n_B$ and the photon number density $n_\gamma$.
Historically these numbers were determined from nucleosynthesis ?????? but modern measurement are based on observations of the cosmic microwave background (CMB) ??????.
A resent analysis ????? finds a value of
\begin{align}
    \eta_B^0 = ?????.
\end{align}



\subsection{Sakharov Conditions}
The \emph{Sakharov Conditions} are a set of three necessary but not sufficient conditions for baryogenesis to happen in the universe \cite{Sakharov_1991}. These are:
\begin{itemize}
    \item Baryon Number has to be violated.
    \item C and CP symmetries have to be broken.
    \item The universe has to be out of equilibrium.
\end{itemize}
Lets proof and discuss them in order.
First, to generate an asymmetry here need to be processes which produce more baryons than anti-baryons i.e. that violate $B$ symmetry.

Second lets consider C and CP violation. They are needed to distinguish between particles and anti-particles. Otherwise here might be processes that produce an asymmetry in the
particle vs anti-particle number but both the process producing more particles and the (conjugated) one that produces more anti-particles happen at the same rate.
Then on average the total number of particles is zero again.
This condition can be formally shown by considering the density matrix $\rho$ of the universe in the Heisenberg picture \footnote{That is the density matrix $\rho$ and all states are time-independent while all observables are time-dependent.}. Its time evolution is governed by the Heisenberg equation
\begin{align}
    i \frac{\mathrm{d} \rho}{\mathrm{d} t} = [H, \rho],
\end{align}
where $H$ is the Hamiltonian of the system.
In order to generate an asymmetry i.e.
\begin{align}
    \langle B \rangle = \Tr [ B \rho ] \neq 0
\end{align}
the total time derivative has to vanish.
We use that the baryon number operator commutes with C and CP transformations
\begin{align}
    C B C^\dagger &= B \\
    (C P) B (C P)^\dagger &= B
\end{align}
as the baryon number and the charge number both flip sign under charge conjugation.
If we assume that C or CP is conserved i.e. (by C/CP we mean C or CP)
\begin{align}
[H, C/CP] = 0,
\end{align}
then
\begin{align}
    \frac{\mathrm{d}}{\mathrm{d} t} \Tr [ B \rho ] &= \Tr [ B \dot{\rho} ] = \Tr [ B [H, \rho] ] \\
    &= \Tr [ (C/CP)^\dagger B (C/CP) [H, \rho] ] = \Tr [ \rho (C/CP)^\dagger B [C/CP, H] ] = 0.
\end{align}
Therefore C and CP violation is required for bayrogenesis. The violation of C and CP symmetry can also be discussed more on the level of the B violating processes. A detailed discussion can be found in \cite[sec 2.3]{Cline:2006ts_Baryogenesis} as well as %\cite[sec. 6.3]{the_early_universe_kolb_and_turner}.

Finally departure from thermal equilibrium is required. This is because in thermal equilibrium the number densities are purely set by the
masses of the particles or anti-particles. If the theory is CPT symmetric, as very QFT with a Lorentz invariant ground-state (vacuum), then the
masses of particles and their anti-particles are the same ??????? \jana{felix an ref fragen}.
Formally lets consider the density matrix in thermal equilibrium
\begin{align}
    \rho = \frac{e^{- \beta H}}{Z},
\end{align}
where $\beta$ is the inverse temperature and $Z$ the partition function.
Then one finds that
\begin{align}
    \Tr [ B \rho ] &=  \Tr [ (CPT) (CPT)^\dagger B \rho ] = \Tr [ (CPT)^\dagger B \rho (CPT) ]  \\
                   &= - \Tr [ B (CPT)^\dagger \rho (CPT) ] = - \Tr [ B \rho ],
\end{align}
since $CPT$ commutes with the Hamiltonian. Therefore $\langle B \rangle = 0$ in this case and one also requires thermal non-equilibrium. This derivation and discussion can also be found in \cite[sec. II]{Trodden:2004mj_baryogenesis_and_leptogenesis}.


\subsection{Sphalerons: Baryon and Lepton Number in the Standard Model}
The baryon number $B$ and the lepton number $L$ are so called accidental
symmetries of the standard model. They are symmetries of the classical lagrangian
and follow directly from the conservation of weak hypercharge \cite[part II.B]{Trodden:2004mj_baryogenesis_and_leptogenesis}.
But this doesn't directly imply that they are also good symmetries of the
quantum theory. Indeed they are not conserved quantities and the four-divergences
of the corresponding Noether-currents are given as \cite[eq. 2.1]{Cline:2006ts_Baryogenesis}
\begin{align}
    \label{eq:B_and_L_current_equations}
    \partial_\mu J^\mu_B = \partial_\mu J^\mu_L = N_f \frac{g_2^2}{32 \pi^2} F^{\mu \nu}_a \tilde{F}_{\mu \nu}^a,
\end{align}
where $N_f$ is the number of fermions, $F$ the SU(2) field strength tensor, $\tilde{F}$ its dual and $g_2$ the SU(2) coupling constant.
One can see how this violation is possible in the quantum theory by considering the path-integral formulation of the theory.
As in the classical theory the lagrangian is invariant under the U(1) transformation of the B and L symmetries but the integration measure of the path-integral is not. This way one could also derive eq. \eqref{eq:B_and_L_current_equations} via the so called Fujikawa-method \cite{Fujikawa_method_PhysRevD.21.2848}.

The right hand side of eq. \eqref{eq:B_and_L_current_equations} is a topological quantity.
A first step in understanding this is to realize that one can write the right hand side as the divergence of the Chern-Simons current \cite[eq. 4.2, 4.3]{Cline:2006ts_Baryogenesis}.
\begin{align}
    \label{eq:chern_simons_current}
    K^\mu = \epsilon^{\mu \alpha \beta \gamma} \left( A^a_\alpha F^a_{\beta \gamma} - \frac{g_2}{3} f^{abc} A^a_\alpha A^b_\beta A^c_\gamma \right),
\end{align}
where $A$ is the SU(2) gauge field and $f^{abc}$ the SU(2) structure constants.
This allows to write the integral of eq. \eqref{eq:B_and_L_current_equations}
over all of space-time as an expression on the bounary of space-time.

\noindent
First we consider vacuum field configurations i.e. field configurations with zero energy density i.e. with vanishing field strength tensor of the SU(2) gauge fields in euclidean space-time. While a vanishing four-potential gives the desired result, any gauge transformed potential yields the same result.
Therefore the gauge fields are given in this case as so called pure gauges \cite[sec. 2.1]{Di_Luzio_2020_Landscape_of_QCD_Axion_models}
\begin{align}
    A_\mu = \frac{i}{g_2} U^{-1} \partial_\mu U.
\end{align}
for some gauge transformation $U(x)$.
Then we find
\begin{align}
    \label{eq:chern_simons_number}
    \frac{g_2^3}{32 \pi^2} \int \mathrm{d}^4 x \, \partial_\mu K^\mu =
    \frac{g_2^3}{32 \pi^2} \int \mathrm{d}^3 \sigma_\mu K_\mu
     =: N_{\mathrm{CS}}.
\end{align}
As SU(2) is topologically equivalent to the tree-sphere $S^3$ and the boundary of euclidean space-time is also the $S^3$, the gauge transformations $U$ are
mappings $S^3 \to S^3$ i.e. the third homotopy group of $S^3$.
It can be shown that this homotopy group is divided into homotopy classes labeled by the value of the Chern-Simons number $N_\mathrm{CW}$ in eq. \eqref{eq:chern_simons_number} which takes integers values (for pure gauge configurations) \cite{homotopy_spheres}.
As the integers are a discrete set, one can not continuously deform a gauge configuration into another one with a different Chern-Simons number.

\noindent
Back in Minkowski space-time we can define the Chern-Simons number of a gauge configuration at time $t$ as \cite[eq. 23]{Dine_2003_Bayrogenesis}
\begin{align}
    N_{\mathrm{CS}} := \int \mathrm{d}^3 x \, K^0
\end{align}
Using the temporal gauge \cite{temporal_gauge_10.1007/BFb0015141} defined by $A^0 = 0$ where $K^i = 0$ we can
write the change of Chern-Simons number between $t_i$ and $t_f$ as
\begin{align}
    \int \mathrm{d}^4 x \partial_\mu K^\mu = \int \mathrm{d} t \, \partial_t \int \mathrm{d}^3 x \, K^0 =
    N_{\mathrm{CS}}(t_f) - N_{\mathrm{CS}}(t_i) = \Delta N_{\mathrm{CS}}(t_i, t_f)
\end{align}
which is an integer if (as in the euclidean case above) the initial and final gauge configurations are pure gauges i.e. vacuum configurations.
This change between vaccua is a quantum tunneling process. Its rate can be computed from instanton configurations which are the solutions to classical equations of motion in euclidean space-time that have finite energy i.e. are localized all dimensions. These solution therefore have to mediate between pure  gauges and change the Chern Simons number by an integer value.
The path integral for the transition rate can be expanded around this instanton solution in euclidean space-time. An introduction to this technique can be found in \cite[Chap. 7]{aspects_of_symmmetry}. The tunneling process itself is also called an instanton.
The resulting amplitude is approximately given by \cite[4.8]{Cline:2006ts_Baryogenesis}
\begin{align}
    e^{-8 \pi^2 / g_2} \approx 10^{-173}
\end{align}
This rate is too low to be observed in any physical process. While it ensures
that matter such our own bodies are stable, it also means that these processes
can't play a role in baryogenesis.
This changes once we consider the system at finite temperature.
Here a change in Chern-Simons number is enhanced by thermal fluctuations called  sphalerons (processes).
Below the electro-weak phase-transition the height of the barrier still
suppresses changes of $N_{\mathrm{CS}}$.
The rate in this phase is given as \cite[eq. 4.10]{Cline:2006ts_Baryogenesis}
\begin{align}
    \Gamma_{\mathrm{sph}} \sim \left( \frac{E_{\mathrm{sph}}}{T} \right)^3 \left( \frac{m_W(T)}{T} \right)^4 T^4 e^{- E_{\mathrm{sph}} / T}
\end{align}
The energy $E_{\mathrm{sph}}$ is the energy of the so called sphaleron configuration. Formally it is a static finite energy solution to the classical
equations of motion that sits on top of the potential barrier between two vaccua (technically the saddle point). In a sense it is the "middle" of the instanton configuration and its free energy represents the height of the barrier between
neighboring vacuua.
It is given by $E_{\mathrm{sph}} \approx 8\pi v / g_2$ with the VEV of the Higgs field $v$ \cite[eq. 4.5]{Cline:2006ts_Baryogenesis}.
A cartoon of the dependence of the energy on the Chern-Simons number and the instanton and sphaleron processes is shown in figure \ref{fig:sphaleron_cartoon}.
\begin{figure}[H]
    \label{fig:sphaleron_cartoon}
    \centering
    \begin{tikzpicture}
    \draw[thick,->] (-4.5,0) -- (4.5,0) node[anchor=west] {Chern-Simons Number};
    \draw[thick,->] (0,0) -- (0,4.5) node[anchor=south] {Field Energy};
    \draw[scale=1, domain=-4.5:4.5, smooth, variable=\x, blue] plot ({\x}, {1.8*(1 - cos(2 * \x r))});
    \foreach \x in {0,3.14,-3.14}
    \filldraw[color=red] (\x, 0) circle (0.1);
    \foreach \x in {3.14/2, -3.14/2}
    \filldraw[color=green] (\x, 1.8*2) circle (0.1);
    \draw[thick,->] (0.0, -0.5) -- (3.14, -0.5);
    \draw[->] (3.3 / 2 + 0.5, 1.8*2 + 0.5 + 0.1) node[anchor=south]  {Sphaleron} -- (3.4 / 2, 1.8*2 + 0.1) ;
    \draw[->] (-2.7, -0.7) node[anchor=north]  {Vacuum Configuration} -- (-3, -0.2) ;
    \node at (1.5, -0.7) {Instanton};
    \draw[->] (-3.14/6, 2) .. controls (-3.14 / 2, 1.8*2 + 1) .. (-3.14 + 3.14/6, 2) node[midway,above,xshift=-8]{Sphaleron Transition};
    \end{tikzpicture}
    \caption{Cartoon of the vacuum structure of a non-albelian gauge theory, which leads to the non-conservation of baryon and lepton numbers in the standard model.
    }
\end{figure}

\noindent
Above the electro-weak phase-transition the vanishing vacuum-expectation value of the Higgs field implies the non-existence of a sphaelron configration and hence a large rate of \cite{sphaleron_rate_symmetric_phase_Moore_2011}
%\cite[]{Cline:2006ts_Baryogenesis}
%\begin{align}
%\Gamma_{\mathrm{sph}} = (1.06 \pm 0.08) \times 10^{- 6} \, T^4
%\end{align}
\begin{align}
    \Gamma_{\mathrm{sph}} &= (0.21 \pm 0.01) \left(\frac{N_c g^2 T}{m_D^2} \right) \left(\ln \left(\frac{m_D}{\gamma} \right) + 3.0410 \right) \frac{N_c^2 - 1}{N_c^2} (N_c \sqrt{2 \pi g})^5 T^4 \\
    \gamma &= \frac{N_c g^2 T}{4 \pi} \left(\ln \left(\frac{m_D}{\gamma}\right) + 3.0410 \right) \\
    m_D^2 &= \frac{2N_c + N_f}{6} g^2 T^2,
\end{align}
where $N_c$ the number of "colors" ($N_c = 2$ for the weak SU(2) theory) and $g = g_2$ the coupling constant.

\noindent
In this regime this process is in equilibrium. From considering the
equilibrium relations between chemical potentials for different charges one
can derive the relation between the $B - L$ and the $B$ charges as \cite[sec 2.5]{Leptogenesis_review_doi:10.1146/annurev.nucl.55.090704.151558}
\begin{align}
    B = \frac{8 N_f + 4 N_H}{22 N_f + 13 N_H} (B - L),
\end{align}
where $N_H$ is the number of Higgs doublets ($N_H = 1$ in the SM). We define the prefactor as the sphaeloron conversion factor $C_\mathrm{sph}$.
This shows an important requirement for a bayrogenesis model.
A baryogenesis process has to produce a $B - L$ asymmetry otherwise the sphalerons
will washout the asymmetry in $B$ if the process happens in the symmetric (un-broken) phase. Since $B - L$ is a conserved quantity in the standard model, physics beyond the standard model is needed.

\subsection{The Weinberg Operator: Effective Lepton-Number Violations}
\label{sec:lapton_number_violations}

In effective field theory one can generate lepton violations at dimension five only from the
Weinberg operator \cite{Weinberg_operator_original_PhysRevLett.43.1566}. It is given by \cite[eq. 2]{Weinberg_at_colliders_Fuks_2021}
\begin{align}
\mathcal{L}_5 = \frac{C_{l l'}}{\Lambda} [H \cdot \bar{L}^c_l] [L_{l'} \cdot H] + h.c.
\end{align}
where $L = (\nu_l,\, l)^T$ is the left handed lepton doublet ($\nu_l$ the left handed neutrino and $l$ the charged lepton) and $H = (h_+,\, h_0)^T$ the Higgs doublet. The index $l$ and $l'$ denote the flavor index and the parameter $C_{l l'}$ encodes a potential flavor dependence.
The superscript $c$ defines charge conjunction, which acts on spinors as $C \psi C = i \gamma^2 \psi^*$
and $\Lambda$ is the energy scale of this operator.
As one can see this operator only contains left-handed leptons and no anti-leptons. Therefore
it generates scattering processes which violate $L$ by $\Delta L = 2$.
At lower energies, below the electro-weak phase-transition where the Higgs field takes a vacuum expectation value, this operators generates a mass-term for left-handed neutrinos of the form
\begin{align}
C_{ll'} \frac{v_{\mathrm{EW}}^2}{\Lambda} C v_l v_l',
\end{align}
where $\langle H \rangle = (0,\, v_{\mathrm{EW}} / \sqrt{2})^T$.
This is a Majorana mass term ????? for the mass matrix $m_{\nu, ll'} = C_{ll'} \frac{v_{\mathrm{EW}}^2}{\Lambda}$
We assume flavor independence and that the sum of neutrino masses
is of order the atmospharic neutrino mass difference $\bar{m}^2 = \sum_i m_{\nu, i} \sim \Delta m_{\mathrm{atm}} = 2.4 \cdot 10^{-3} \, \mathrm{eV}^2$ as in \cite{Kusenko_2015_Axion_Leptogenesis}.
Then the effective cross-section for $\Delta L = 2$ processes
is given by \cite[eq. 9]{Kusenko_2015_Axion_Leptogenesis}
\begin{align}
\sigma_{\Delta L = 2} \approx \frac{3}{32 \pi^2} \frac{\bar{m}^2}{v_{\mathrm{EW}}^4}.
\end{align}
This approximation is valid if the center-of-mass energies in the processes are always small compared to the energy scale $\Lambda$.
There are different possible UV completions that would generate the Weinberg operator
One is the see-saw mechanism which introduces heavy right handed neutrinos and explains the low masses of the left-handed neutrinos.
In this case the Weinberg operator is mediated by right handed neutrinos and
couples to the left-handed lepton and the Higgs doublets via Yukawa couplings.
If one integrates out the right handed neutrino by assuming that their mass is
large compared to the center-of-mass energy (essentially neglecting the momentum in the denominator of the internal propagator), one obtains the Weinberg operator.
A cartoon of this is shown in figure ?????.
For this approximation to be valid in our computations, we require that
the masses of the right-handed neutrinos are large compared to the temperatures
of the thermal plasma. Then the center-of-mass energy of the scatterings are
small compared to the mass of the mediating right-handed neutrinos.
Also this ensures that the right-handed neutrinos are not produced in the plasma and therefore there is not interference with other mechanisms for leptogenesis.



\subsection{Baryogenesis Models}
In this section we will briefly describe several possible models for baryogenesis but will go into detail for any of them.
We give a source that serves as a pedagogical introduction for each of them.

\begin{itemize}
    \item The oldest mechanism for baryogenesis is baryogenesis in grand unified theories (GUT-BG).
    For instance in an SU(5) GUT all requirements for successful baryogenesis are
    fulfilled. This theory contains gauge bosons $X^\mu$ and Higgs bosons $Y$ with decays that directly
    violate baryon and/or lepton number. Given appropriate complex phases in the Yukawa couplings
    their decays violate CP. The decay of particles is a non-equilibrium process. Hence the Sakharov conditions are satisfied and indeed successful baryogenesis can be realized as detailed computations show.
    Unfortunately this mechanism requires a reheating temperature that is too high to avoid other constrains. For a detailed discussion see ref. \cite[sec. 3]{Dine_2003_Bayrogenesis}.
    \item Another mechanism for baryogenesis is electro-weak baryogenesis (EW-BG).
    This model implements baryogenesis at the electro-weak phase transition.
    Above symmetry breaking the net baryon number is zero.
    At the phase transition bubbles where the electro-weak symmetry $SU(2)_W \times U(1)_Y$ is broken might form, which then
    move through the plasma.
    Scatterings with the bubble wall create an asymmetry in CP and C in front of the bubble if C and CP violating interactions are present.
    Those asymmetries bias the sphaleron rate to create more particles than anti-particles in the unbroken phase.
    Some particles then are collected by the bubble. Inside the bubble the sphaleron rate is suppressed and the
    baryon number is not washed-out.
    The formation of bubbles only happens if the elector-weak phase-transition is first order.
    Unfortunately this is only the case in the standard model if the mass of the Higgs boson is much lower than
    the measured value.
    This property is modified if additional particles are introduced that then modify the effective finite-temperature
    Higgs-potential that determines whether the phase-transition is first order.
    For instance the minimal supersymmetric standard model (MSSM) would allow for this mechanism to happen.
    One appealing aspect of the theory is that the additional couplings need to appear at energies expressible to future collider
    experiments. An overview over this field is given in \cite{Electroweak_baryogenesis_Morrissey_2012}.
    \item Leptogenesis is a mechanism that requires comparatively little additional assumptions beyond the standard model.
    A natural way to explain the small neutrino masses in the standard model is to introduce heavy right handed neutrinos via the see-saw mechanism.
    Those neutrinos could be produced in the early universe and then decay into leptons violating lepton number as well as C and CP. No particles carrying baryon
    number take part in this process.
    Therefore a $B - L$ asymmetry is generated that is converted into baryons by the sphalerons. An introduction to the model is given in \cite{Leptogenesis_review_doi:10.1146/annurev.nucl.55.090704.151558} and \cite{Leptogenesis_Buchmüller:2014} while more technical details are discussed in \cite{Pedestrians_Buchm_ller_2005}.
    \item The last model we will discuss in this section is Affleck-Dine baryogenesis (AD-BG).
    In this mechanism a scalar field carrying baryon number is introduced (such as in a super-symmetric model).
    This scalar field has a cosmological evolution leading to a high occupation number. This scalar field then decays into standard model particles, creating the baryon asymmetry today.
    This model works at lower energy-scales and it can also explain why the abundance of baryonic and dark matter is so similar. An introduction to AD-BG can be found in \cite[Part III]{Dine_2003_Bayrogenesis}.
\end{itemize}


\section{Spontaneous Leptogenesis with Axions}

\subsection{Spontaneous Baryogenesis}
\label{sec:spontaneour_bayrogenesis}

All the mechanism above mentioned fullfill the Sakharov conditions.

In ???? ???? and ???? pointed out that there is a loop hole in the Sakharov conditions
and that is that they assume the theory to be invariant under CPT.
In general any QFT with a Lorentz invariant groundstate is CPT symmetric but in the early universe the ground state of the theory not Lorentz invariant due to being at finite temperature and more importantly \footnote{Some thing about curved backgrounds???????}
due to the expanding background. In particular the expectation value of a scalar field and its time derivtive will not vanish in general. A coupling to such a field such as ???????
\begin{align}
    \mathcal{L} \ni - \frac{1}{f} \phi \partial_\mu J^\mu_B
\end{align}
violates CPT. Here $\phi$ is the scalar field, $f$ some energy scale of this new physics and $J_B$ the Baryon current.
By partial integration and by assuming that the field $\phi$ is homogenious one finds
\begin{align}
    \frac{1}{f} J_B^\mu \partial_\mu \phi = \frac{1}{f} J_B^0 \dot{\phi}
    %= \frac{1}{f} (n_b - n_{\bar{b}}) \dot{\phi}
    = \frac{1}{f} n_B \dot{\phi},
\end{align}
where $n_b$ is the number density of the baryons and $n_{\bar{b}}$ of the anti-baryons respectively.
If one then compares this term with the density matrix of the grand-canoncal ensemble
\begin{align}
    \mathcal{Z} = e^{-\beta H + \beta \mu N}
\end{align}
one finds that the chemical potential $\mu$ is modified by this
interaction as
\begin{align}
    \mu \to \mu_\mathrm{eff} = \mu + \frac{\phi}{f}
\end{align}
If the baryon asymmetry is not protected i.e. there exist B-violating processes in the thermal plasma, the baryon asymmetry is then given in equilibrium as ?????
\begin{align}
    n_B^\mathrm{eq} = \mu_\mathrm{eff} \frac{T^2}{6}
\end{align}
In this case the final baryon asymmetry is given by the value of $n^\mathrm{eq}_B$ at the freeze-out of the baryon number violating interaction.
This mechanism indeed does not require C and CP violation because the baies of the interacion towards baryons comes from the chemical potetials and not from the cross-sections of the interactions.
Note that still C and CP violation are allowed to occur.
Furhtermore this mechanism also works in equilibrium.
But in later cases we need to study the case where the interaction is only close to equilibrium.

\noindent
What kind of field could $\phi$ be?

\jana{
\begin{itemize}
    \item original idea
    \item violation of CPT, why okay?
\end{itemize}
}












\subsection{Axions}
\label{sec:axions}

One potentially cosmologically relevant type of scalar fields are axion-like fields.
Axions were first introduced as a solution to the strong CP problem.
In QCD the operator
\begin{align}
    \mathrm{Tr} G_{\mu \nu} 2\tilde{G}^{\mu \nu},
\end{align}
where $G = ?????$ is the gluon field strength tensor,
has similar properties as the equivanlent term for the $SU(2)$ field appearing
in ???????. But there it is not connected to a symmetry and therefore can't
be removed from the lagagian ????? TODO: make sure that I understand this correctly
Therefore a new term has to appear in the lagragian
\begin{align}
    \mathcal{L}
\end{align}
In princple the parameter $\theta$ can take values betwenn 0 and $2\pi$ but measurements of the electric dipol moment of the neutron constrain $\theta$ to
be $\theta < ????$ ?????.
This fine tuning problem is the strong CP problem.
The axion solution to this problem is to introduce a new pseudo-scalar field
$\theta$, the axion field that couples as ????
\begin{align}
    ???????
\end{align}
and therefore corrects the theta parameter. The ?????? theorem ensurs that
the mimium of this potetial is at the CP conserving minimum.
This coupling comes from the field rotation from a couping to fermions which in turn comes from
the coupling of the UV completion of the axion theory.
The axion is a pseudo-Nambu-Goldstone boson. Its dynamics are derived from
Gnerically we call all pseudo scalar bosons that couple to gauge fields via anaomaly term as in eq. ????.
axion-like particles (APLs).
The effective potetial for the axion field is given by the dependence of QCD vacuum energy
on the theta parameter $\theta$ and is given (in the one-instanton gas approximation) as ??????
\begin{align}
    V(\theta; T) &= m_a^2(T) (1 - \cos(\theta)) \\
    m_a &= m_a(T = 0) \cdot \min\left(1, \left( T / \Lambda \right)^p \right)
\end{align}
Cosmologically an axion field is in a coherent state that can be described by classical field equations.
The effective axion action is given by
\begin{align}
    1 = 1
\end{align}
Varying the action yields the equations of motion
\begin{align}
    \ddot{a} + 3 H \dot{a} + \frac{\partial V(a; T)}{\partial a} = 0
\end{align}
The most well known application of axions is that they could be the dark matter since
the coherent oscillations of the axion field have the equation of state of matter.
Typically one assumes the realignment mechanism, where the initial kinetic energy is not large enough to overcome the potential barrier of the axion potential. In this case the initial motion of the axion is damped and the field is frozen in (is approximately constant in time). Then only the initial value of the field is relevant.
When the Hubble friction no longer dominates the axion mass i.e.
\begin{align}
    3H(T_\mathrm{osc}) = m_a(T_\mathrm{osc}),
\end{align}
the field starts to oscillate.
Then the co-moving number density is conserved and the final dark matter abundance is given by
\begin{align}
    \Omega_a h^2 = ???????
\end{align}
This means at very high masses the relic abundance is too low to be a relevant dark matter component.

In quantum chromodynamics (QCD) the vacuum structure is non trivial.
????
In general axion like particles (APLs) are pseudo scalar bosons that couple
to gauge field via an anomaly term.
Axions are pseudo scalar bosons


\jana{
\begin{itemize}
    \item strong cp problem
    \item axion solution
    \item axion like particles
    \item efftive couplings to gauge bosons and ferminons
    \item decay rate into gauge bosons (like pion)
    \item cosmological axion field

\end{itemize}
}

\subsection{Reheating}
The violation of lepton number comes from the Weinberg operator (discussed in section ????).
The temperature scale for our model is routhly given by the equilibration temperature of the Weinberg operator i.e.
\begin{align}
    \Gamma_W(T_\mathrm{eq}) = H(T_\mathrm{eq}) \Rightarrow T \approx 10^{13} \, \mathrm{GeV}
\end{align}
This means this process has to happen during/directly after reheating. The reheating process is described by the energy rate equations and the Friedmann equation ????????
\begin{align}
    \dot{\rho_\phi} + 3 H \rho_\phi &= - \Gamma_\phi \rho_\phi \\
    \dot{\rho_R} + 4 H \rho_R &= + \Gamma_\phi \rho_\phi \\
    3 H^2 M_\mathrm{pl}^2 &= \rho_\mathrm{tot} \approx \rho_\phi + \rho_R,
\end{align}
where $\rho_\phi$ is the energy density of the inflaton field, $\rho_R$ the radiation energy density and $\Gamma_\phi$ the decay rate of the inflaton to SM particles.
During this epoch we describe the motion of the axion field using eq. ?????????? with given the Hubble parameter.
For simplicity we assume that the axion has a constant fixed mass, that we treed as an open parameter. We assume the standard relaighment scenario. Since the oscillation has to start at very high values of the Hubble parameter, the mass of the axion has to be high as well (see eq. ???).
This means that this axion can't be the QCD axion. Its mass is assumed to come from some hidden sector.
We then assume that the axion directly couples to the lepton current like in eq. ?????. Then effective chemical potential is given by $\mu_\mathrm{eff} = \dot{a}/f_a$ according to eq. ????.
The process of creating the asymmetry is described by the Boltzmann equation ??????
\begin{align}
    \dot{n_L} + 3 H n_L = - \Gamma_L(T) (n_L - n_L^\mathrm{eq}(\mu_\mathrm{eff}, T))
\end{align}
with
\begin{align}
    \rho_R = \frac{\pi^2}{30} g_{*}(T) T^4
\end{align}
\begin{align}
    n_L^\mathrm{eq} =
\end{align}
\begin{align}
    \Gamma_L =
\end{align}
We assume that there is no initial adunance of letpon asymmetry i.e. $n_L = 0$ at the end of inflation.
The
This set of differential equations can then be solved numerically. The evolution of projected final baryon asymmetry (from eq. ?????) is shown in figure ????? for two different values of the inflaton decay rate $\Gamma_\phi$ and the axion mass $m_a$.

Furthermore one has to take the decay of the axion into account.
Since it couples the lepton current, it also couples to the

\subsection{Discssion of the Mechanism}
Let us now discuss the mechanism and limit on the parameters in particular.
The parameter of this model are $(\theta_i, m_a, f_a, H_\mathrm{inf}, \Gamma_\phi)$.
For simplicity we assume

\subsection{The Problem of Field Basis and the Transport Equation}
The model we discussed so far is based on ref. ??????.
There the authors argumeted that the coupling originates from a field redefinition
\begin{align}
    ???????
\end{align}
of the anaomaly coupling of the axion to the SU(2) gauge field (as in eq. ??????).
As pointed out in ref. ???? and ref. ???????, this simultaneously induces an energy shift
?????????????????????????????????????ß
?????????????????????????????????
Therefore

\begin{align}
    \dot{q}_i + 3 H q_i = - \sum_\alpha \Gamma_\alpha n^\alpha_i \Big[
    \sum_j n^\alpha_j \left( \frac{\mu_j}{T} \right) - n_S^\alpha \frac{\dot{a} / f_a}{T} \Big]
\end{align}


\jana{
\begin{itemize}
    \item coupling of axion derivative to lepton current -> effective chemical potential
    \item $\Delta L = 2$ processes and the efftive crossection
    \item Boltzmann equation of the asymmetry
    \item required temperature
    \item reheating (energy + Friedmann eq.)
    \item decay of axions via the anomaly
    \item present solution
    \item decay of inflaton injects entropy $\rightarrow$ dilution
    \item decay of axion injects entropy $\rightarrow$ dilution
    \item washout from oscillations
    \item analytical solution
    \item a-priori constrains
    \item dilution frmo axion ineffctive at low $f_a$
    \item apriori constrains lead to minimal $m_a, \Gamma_\phi$
\end{itemize}
}

\subsection{Outlook}
\label{sec:outlook}

Starting from the foundation established above, we can processed in several directions.
In this section we briefly discuss each. The goal of this project is to explore
different models that can successfully implement baryogenesis, while avoiding cosmological and
experimental constrains.
In particular we are interested in models that also implement a viable dark matter candidate
and/or are in regions of the parameter that, while not excluded right now, allow for realistic future experimental searches.













\section{Conclusion}
\label{sec:conclusion}
\jana{TODO!!!!!}


\newpage
\printbibliography
\end{document}

