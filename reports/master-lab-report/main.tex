\documentclass[13pt,a4paper,titlepage]{article}
%\documentclass[13pt,a4paper,twoside,titlepage]{article}
\usepackage[ngerman, english]{babel}
\usepackage[utf8]{inputenc}
\usepackage{latexsym,amssymb,amsmath,amstext}
\usepackage{graphicx}
\usepackage{float}
\usepackage[backend=bibtex, sorting=none]{biblatex}
\usepackage{amsmath}
\usepackage{ dsfont }
\usepackage{csquotes}
\usepackage{slashed}
\usepackage{subfigure}
\usepackage{listings}
\usepackage{subfigure}
\usepackage{verbatim}
\usepackage{xcolor}
\usepackage{multicol}
\usepackage{feynmf}
\usepackage{abstract}
\usepackage{bbold}
\usepackage[a4paper, left=2cm, right=2cm, top=2cm, bottom=3cm, bindingoffset=10mm]{geometry}
\usepackage[compat=1.1.0]{tikz-feynman}
%\usepackage[onehalfspacing]{setspace}
\usepackage{tikz}
\usepackage{pgfplots}
\lstset{
  breaklines=true,
  postbreak=\mbox{\textcolor{red}{$\hookrightarrow$}\space},
}

\newcommand{\Tr}{\operatorname{Tr}}
\newcommand{\note}[1]{{\color{red}[\textbf{#1}]}}
\newcommand{\diff}{\ensuremath{\mathrm{d}}}

\bibliography{refs}

\begin{document}
%\pagenumbering{gobble}
\begin{titlepage}

\vspace*{3cm}

\centering
\textsc{\Large Masters Project - Lab Report}
% \textsc{\Large \bfseriedvanced Masters Project - Lab Report}
\\
\vspace*{1cm}
%\rule{\textwidth}{1pt}\\[0.1cm]
{\Large \bfseries
Spontaneous Baryogenesis using Axions
\\[0.1cm]}
%\rule{\textwidth}{1pt}
\vspace*{1cm}
{
\centering\Large
prepared by\\[0.2cm]
{\bfseries Jana Riess}\\[0.2cm]
}

\vspace*{1cm}
{
    \centering\Large
    advised by\\[0.2cm]
    {\bfseries Prof. Dr. Laura Covi}\\[0.2cm]
}

\vspace*{1cm}
{
    \centering\Large
    co-advised by\\[0.2cm]
    {\bfseries Prof. Dr. David J. E. Marsh}\\[0.2cm]
}

\vspace*{1cm}
{
    \centering\Large
    at the\\[0.2cm]
    {\bfseries Institute for Theoretical Physics \\ George August Universität Göttingen}\\[0.2cm]
}

\vspace*{1cm}
{
    \centering\Large
    Date of Submission:\\[0.2cm]
    {\bfseries 30.7.2020}\\[0.2cm]
}




\vspace*{2cm}

\begin{Large}
\begin{tabular}{ll}
%\bfseries Date of Submission: &21.01.2020\\
\end{tabular}
\end{Large}

\vspace*{1.5cm}

\end{titlepage}

\tableofcontents
\newpage

\newcommand{\jana}[1]{{\color{magenta}{#1}}}


\section{Introduction}
\label{sec:introduction}

\noindent
The baryon asymmetry of the universe (BAU) is a precondition for life to exist in the universe.
Yet its origin is unknown and a challenge in current physics research to find different possible models
that can explain this asymmetry.
Since the discovery of anti-particles it has been a problem to understand why here should be
an asymmetry between particles and their anti-partners as most simple theories are symmetric under
exchange of particles and anti-particles including some that are
very successful in describing nature. Moreover we also need a mechanism that can
produce the asymmetry dynamically as any initial asymmetry would have diluted away during inflation.

\noindent
This report is organized as follows: It is split into two parts.
In the first part we discuss the basics of baryogenesis (sec. \ref{sec:basics_of_baryogenesis}).
This includes parts like baryon number violations in the standard model from sphalerons in section
\ref{sec:sphaelrons} and the Weinberg operator in section \ref{sec:lepton_number_violations} that will
be used in the second part as well as general considerations like the Sakharov conditions in sec. \ref{sec:sakharov_conditions} and an overview over different baryogenesis models in sec. \ref{sec:baryogenesis_models}.
In the second part (sec. \ref{sec:spontaneour_bayrogenesis}) we will discuss the model that will be topic of this
thesis project, namely spontaneous baryogenesis and its implementation with axions.
This involves as the main parts the discussion of the mechanism and the computations involved in sec. \ref{sec:spontaneour_bayrogenesis_tradiaonal} and \ref{sec:model}.
As well as some background about axions (sec. \ref{sec:axions}) and the reheating epoch (\ref{sec:reheating}).

\section{Basics of Baryogenesis}
\label{sec:basics_of_baryogenesis}

This section reviews the fundamentals of baryogenesis.

\subsection{Evolution of the Baryon Asymmetry}
\label{sec:evolution_of_the_baryon_asymmetry}
First let us review the evolution of the baryon density. We know from collider experiments that at energies below $\sim 1 \, \mathrm{TeV}$ the baryon and lepton numbers are conserved.
Therefore the process of baryogenesis had to happen at higher energies.
At these energies the baryons/leptons and photons are in chemical equilibrium.
This means that from thermodynamics we know that
\begin{align}
\label{eq:baryon_equilibrium}
n_b + n_{\bar{b}} \approx n_\gamma,
\end{align}
where $n_b, n_{\bar{b}}, n_\gamma$ are the number densities of baryons, anti-baryons and photons.
At lower energies the baryons annihilate into photons. That is all except the baryon excess $n_B = n_b - n_{\bar{b}}$. Now the baryon asymmetry $n_B$ is maximal. Those remaining baryons make up the baryon density today.
%We want to quantify the amount of asymmetry between baryons and anti-baryons.
%Today the asymmetry is maximal but
%above temperatures of $\sim 1 \, \mathrm{GeV}$ baryons and anti-baryons are in chemical equilibrium.
Because of eq. \eqref{eq:baryon_equilibrium} it is convenient to express the asymmetry as
the ratio between the difference between the number densities of baryons $n_b$ and anti-baryons $n_{\hat{b}}$
over the number densities of photons \cite[eq. 3.52]{the_early_universe_kolb_and_turner}
\begin{align}
    n_\gamma = 2 \frac{\zeta(3)}{\pi^2} T^3
\end{align}
i.e. as the asymmetry parameter \cite[eq. 1.2]{Cline:2006ts_Baryogenesis}
\begin{align}
    \eta_B = \frac{n_B}{n_\gamma}.
\end{align}
This quantity is conserved as the universe expands because both the baryon number density
and the photon density scale as $\sim 1/R^3 \sim T^3$.

However this is not a conserved quantity if the co-moving number of photons changes due to annihilation
of particles e.g. baryons or leptons into photons.
Since the total co-moving entropy is conserved (the entropy of the annihilating particles is converted into the photons), the baryon abundance
\begin{align}
    \label{eq:baryon_abundance}
    Y_B := \frac{n_B}{s},
\end{align}
where $s$ is the entropy density
\begin{align}
    s = \frac{2 \pi^2}{45} g_{*, s}(T) T^3
\end{align}
with the effective relativistic degrees of freedom for the entropy density $g_{*, s}(T)$
is conserved since both scale as $\sim T^3$.
Alternatively one can compute the present day baryon asymmetry $\eta_B^0$ from eq. \eqref{eq:baryon_abundance} by considering the ratio
\begin{align}
    \frac{\eta_B}{Y_B} = \frac{\pi^4}{45 \zeta(3)} g_{*, s}(T)
\end{align}
leading to
\begin{align}
    \label{eq:asymmetry_redshift}
    \eta_B^0 = \frac{g_{*,s}^0}{g_{*, s}^{\gg 1 \, \mathrm{Gev}}} \eta_B^{\gg 1 \, \mathrm{Gev}}.
\end{align}
where in the standard model one has $g_{*,s}^0 = 43/11$ and $g_{*, s}^{\gg 1 \, \mathrm{Gev}} = 427/4$ \cite[sec. 3.4]{the_early_universe_kolb_and_turner}.

\noindent
Historically the numbers were determined from big bang nucleosynthesis (BBN) \cite[sec. III.1]{Matter_and_Antimatter_Canetti_2012} but modern measurement are based on observations of the cosmic microwave background (CMB).
One finds a value of \cite[eq. 7]{Matter_and_Antimatter_Canetti_2012}
\begin{align}
    \label{eq:observed_bau}
    \eta_B^\mathrm{obs} = 6.160^{+ 0.153}_{-0.156} \cdot 10^{-10}
\end{align}
for the observed baryon asymmetry.

\subsection{Sakharov Conditions}
\label{sec:sakharov_conditions}

The \emph{Sakharov Conditions} are a set of three necessary but not sufficient conditions for baryogenesis to happen in the universe \cite{Sakharov_1991}. These are:
\begin{itemize}
    \item Baryon Number has to be violated.
    \item C and CP symmetries have to be broken.
    \item The universe has to be out of equilibrium.
\end{itemize}
Lets proof and discuss them in order.
First, to generate an asymmetry here need to be processes which produce more baryons than anti-baryons i.e. that violate $B$ symmetry.

\noindent
Second let us consider C and CP violation. They are needed to distinguish between particles and anti-particles. Otherwise there might be processes that produce an asymmetry in the
particle vs anti-particle number but both the process producing more particles and the (conjugated) one that produces more anti-particles happen at the same rate.
Then on average the total number of particles is zero again.
This condition can be formally shown by considering the density matrix $\rho$ of the universe in the Heisenberg picture \footnote{That is the density matrix $\rho$ and all states are time-independent while all observables are time-dependent.}. Its time evolution is governed by the Heisenberg equation
\begin{align}
    i \frac{\mathrm{d} \rho}{\mathrm{d} t} = [H, \rho],
\end{align}
where $H$ is the Hamiltonian of the system.
In order to generate an asymmetry i.e.
\begin{align}
    \langle B \rangle = \Tr [ B \rho ] \neq 0
\end{align}
the total time derivative has to vanish.
We use that the baryon number operator commutes with C and CP transformations
\begin{align}
    C B C^\dagger &= B \\
    (C P) B (C P)^\dagger &= B
\end{align}
as the baryon number and the charge number both flip sign under charge conjugation.
If we assume that C or CP is conserved i.e. (by C/CP we mean C or CP)
\begin{align}
[H, C/CP] = 0,
\end{align}
then
\begin{align}
    \frac{\mathrm{d}}{\mathrm{d} t} \Tr [ B \rho ] &= \Tr [ B \dot{\rho} ] = \Tr [ B [H, \rho] ] \\
    &= \Tr [ (C/CP)^\dagger B (C/CP) [H, \rho] ] = \Tr [ \rho (C/CP)^\dagger B [C/CP, H] ] = 0.
\end{align}
Therefore C and CP violation is required for bayrogenesis. The violation of C and CP symmetry can also be discussed more on the level of the B violating processes. A detailed discussion can be found in \cite[sec 2.3]{Cline:2006ts_Baryogenesis}. % as well as %\cite[sec. 6.3]{the_early_universe_kolb_and_turner}.

\noindent
Finally departure from thermal equilibrium is required. This is because in thermal equilibrium the number densities are set by the
masses of the particles or anti-particles. If the theory is CPT symmetric, as very QFT with a Lorentz invariant ground-state (vacuum), then the
masses of particles and their anti-particles are the same. % ??????? \jana{felix an ref fragen}.
Formally lets consider the density matrix in thermal equilibrium
\begin{align}
    \rho = \frac{e^{- \beta H}}{Z},
\end{align}
where $\beta$ is the inverse temperature and $Z$ the partition function.
Then one finds that
\begin{align}
    \Tr [ B \rho ] &=  \Tr [ (CPT) (CPT)^\dagger B \rho ] = \Tr [ (CPT)^\dagger B \rho (CPT) ]  \\
                   &= - \Tr [ B (CPT)^\dagger \rho (CPT) ] = - \Tr [ B \rho ],
\end{align}
since $CPT$ commutes with the Hamiltonian. Therefore $\langle B \rangle = 0$ in this case and one also requires thermal non-equilibrium. This derivation and discussion can also be found in \cite[sec. II]{Trodden:2004mj_baryogenesis_and_leptogenesis}.


\subsection{Sphalerons: Baryon and Lepton Number in the Standard Model}
\label{sec:sphaelrons}

The baryon number $B$ and the lepton number $L$ are so called accidental
symmetries of the standard model. They are symmetries of the classical lagrangian
\cite[part II.B]{Trodden:2004mj_baryogenesis_and_leptogenesis}.
But this doesn't directly imply that they are also good symmetries of the
quantum theory. Indeed they are not conserved quantities and the four-divergences
of the corresponding Noether-currents are given as \cite[eq. 2.1]{Cline:2006ts_Baryogenesis}
\begin{align}
    \label{eq:B_and_L_current_equations}
    \partial_\mu J^\mu_B = \partial_\mu J^\mu_L = N_f \frac{g_2^2}{32 \pi^2} F^{\mu \nu}_a \tilde{F}_{\mu \nu}^a,
\end{align}
where $N_f$ is the number of fermions, $F$ the SU(2) field strength tensor, $\tilde{F}$ its dual and $g_2$ the SU(2) coupling constant.
One can see how this violation is possible in the quantum theory by considering the path-integral formulation of the theory.
As in the classical theory the lagrangian is invariant under the U(1) transformation of the B and L symmetries but the integration measure of the path-integral is not. This way one could also derive eq. \eqref{eq:B_and_L_current_equations} via the so called Fujikawa-method \cite{Fujikawa_method_PhysRevD.21.2848}.

The right hand side of eq. \eqref{eq:B_and_L_current_equations} is a topological quantity.
A first step in understanding this is to realize that one can write the right hand side as the divergence of the Chern-Simons current \cite[eq. 4.2, 4.3]{Cline:2006ts_Baryogenesis}.
\begin{align}
    \label{eq:chern_simons_current}
    K^\mu = \epsilon^{\mu \alpha \beta \gamma} \left( A^a_\alpha F^a_{\beta \gamma} - \frac{g_2}{3} f^{abc} A^a_\alpha A^b_\beta A^c_\gamma \right),
\end{align}
where $A$ is the SU(2) gauge field and $f^{abc}$ the SU(2) structure constants.
This allows to write the integral of eq. \eqref{eq:B_and_L_current_equations}
over all of space-time as an expression on the bounary of space-time.

\noindent
First we consider vacuum field configurations i.e. field configurations with zero energy density i.e. with vanishing field strength tensor of the SU(2) gauge fields in euclidean space-time. While a vanishing four-potential gives the desired result, any gauge transformed potential yields the same result.
Therefore the gauge fields are given in this case as so called pure gauges \cite[sec. 2.1]{Di_Luzio_2020_Landscape_of_QCD_Axion_models}
\begin{align}
    A_\mu = \frac{i}{g_2} U^{-1} \partial_\mu U
\end{align}
for some gauge transformation $U(x)$.
Then we find
\begin{align}
    \label{eq:chern_simons_number}
    \frac{g_2^3}{32 \pi^2} \int \mathrm{d}^4 x \, \partial_\mu K^\mu =
    \frac{g_2^3}{32 \pi^2} \int \mathrm{d}^3 \sigma_\mu K_\mu
     =: N_{\mathrm{CS}}.
\end{align}
As SU(2) is topologically equivalent to the tree-sphere $S^3$ and the boundary of euclidean space-time is also the $S^3$, the gauge transformations $U$ are
mappings $S^3 \to S^3$ i.e. the third homotopy group of $S^3$.
It can be shown that this homotopy group is divided into homotopy classes labeled by the value of the Chern-Simons number $N_\mathrm{CW}$ in eq. \eqref{eq:chern_simons_number} which takes integers values (for pure gauge configurations) \cite{homotopy_spheres}.
As the integers are a discrete set, one can not continuously deform a gauge configuration into another one with a different Chern-Simons number.

\noindent
Back in Minkowski space-time we can define the Chern-Simons number of a gauge configuration at time $t$ as \cite[eq. 23]{Dine_2003_Bayrogenesis}
\begin{align}
    N_{\mathrm{CS}} := \int \mathrm{d}^3 x \, K^0
\end{align}
Using the temporal gauge \cite{temporal_gauge_10.1007/BFb0015141} defined by $A^0 = 0$ where $K^i = 0$ we can
write the change of Chern-Simons number between $t_i$ and $t_f$ as
\begin{align}
    \int \mathrm{d}^4 x \partial_\mu K^\mu = \int \mathrm{d} t \, \partial_t \int \mathrm{d}^3 x \, K^0 =
    N_{\mathrm{CS}}(t_f) - N_{\mathrm{CS}}(t_i) = \Delta N_{\mathrm{CS}}(t_i, t_f)
\end{align}
which is an integer if (as in the euclidean case above) the initial and final gauge configurations are pure gauges i.e. vacuum configurations.
This change between vaccua is a quantum tunneling process. Its rate can be computed from instanton configurations which are the solutions to classical equations of motion in euclidean space-time that have finite energy i.e. are localized all dimensions. These solution therefore have to mediate between pure  gauges and change the Chern Simons number by an integer value.
The path integral for the transition rate can be expanded around this instanton solution in euclidean space-time. An introduction to this technique can be found in \cite[Chap. 7]{aspects_of_symmmetry}. The tunneling process itself is also called an instanton.
The resulting amplitude is approximately given by \cite[4.8]{Cline:2006ts_Baryogenesis}
\begin{align}
    e^{-8 \pi^2 / g_2} \approx 10^{-173}.
\end{align}
This rate is too low to be observed in any physical process. While it ensures
that matter like our own bodies are stable, it also means that these processes
can't play a role in baryogenesis.
This changes once we consider the system at finite temperature.
Here a change in Chern-Simons number is enhanced by thermal fluctuations called sphalerons (processes).
Below the electro-weak phase-transition the height of the potential barrier still
suppresses changes of $N_{\mathrm{CS}}$.
The rate in this phase is given as \cite[eq. 4.10]{Cline:2006ts_Baryogenesis}
\begin{align}
    \Gamma_{\mathrm{sph}} \sim \left( \frac{E_{\mathrm{sph}}}{T} \right)^3 \left( \frac{m_W(T)}{T} \right)^4 T^4 e^{- E_{\mathrm{sph}} / T}.
\end{align}
The energy $E_{\mathrm{sph}}$ is the energy of the so called sphaleron configuration. Formally it is a static finite energy solution to the classical
equations of motion that sits on top of the potential barrier between two vaccua (technically the saddle point). In a sense it is the "middle" of the instanton configuration and its free energy represents the height of the barrier between
neighboring vacua.
It is given by $E_{\mathrm{sph}} \approx 8\pi v / g_2$ with the VEV of the Higgs field $v$ \cite[eq. 4.5]{Cline:2006ts_Baryogenesis}.
A cartoon of the dependence of the energy on the Chern-Simons number and the instanton and sphaleron processes is shown in figure \ref{fig:sphaleron_cartoon}.
\begin{figure}[H]
    \label{fig:sphaleron_cartoon}
    \centering
    \begin{tikzpicture}
    \draw[thick,->] (-4.5,0) -- (4.5,0) node[anchor=west] {Chern-Simons Number};
    \draw[thick,->] (0,0) -- (0,4.5) node[anchor=south] {Field Energy};
    \draw[scale=1, domain=-4.5:4.5, smooth, variable=\x, blue] plot ({\x}, {1.8*(1 - cos(2 * \x r))});
    \foreach \x in {0,3.14,-3.14}
    \filldraw[color=red] (\x, 0) circle (0.1);
    \foreach \x in {3.14/2, -3.14/2}
    \filldraw[color=green] (\x, 1.8*2) circle (0.1);
    \draw[thick,->] (0.0, -0.5) -- (3.14, -0.5);
    \draw[->] (3.3 / 2 + 0.5, 1.8*2 + 0.5 + 0.1) node[anchor=south]  {Sphaleron} -- (3.4 / 2, 1.8*2 + 0.1) ;
    \draw[->] (-2.7, -0.7) node[anchor=north]  {Vacuum Configuration} -- (-3, -0.2) ;
    \node at (1.5, -0.7) {Instanton};
    \draw[->] (-3.14/6, 2) .. controls (-3.14 / 2, 1.8*2 + 1) .. (-3.14 + 3.14/6, 2) node[midway,above,xshift=-8]{Sphaleron Transition};
    \end{tikzpicture}
    \caption{Cartoon of the vacuum structure of a non-albelian gauge theory, which leads to the non-conservation of baryon and lepton numbers in the standard model.
    On the x-axis the Chern-Simons number is shown and on the y-axis the free-energy of the gauge field.
    The energy is sketched as a blue line. The vacuum-configurations are indicated by red dots and the sphaleron configurations by green ones. Sphaleron and instanton transitions are indicated by arrows from one vacuum state
    to the next.
    }
\end{figure}

\noindent
Above the electro-weak phase-transition the vanishing vacuum-expectation value of the Higgs field implies the non-existence of a sphaelron configration and hence a large rate of \cite{sphaleron_rate_symmetric_phase_Moore_2011}
%\cite[]{Cline:2006ts_Baryogenesis}
%\begin{align}
%\Gamma_{\mathrm{sph}} = (1.06 \pm 0.08) \times 10^{- 6} \, T^4
%\end{align}
\begin{align}
    \Gamma_{\mathrm{sph}} &= (0.21 \pm 0.01) \left(\frac{N_c g^2 T}{m_D^2} \right) \left(\ln \left(\frac{m_D}{\gamma} \right) + 3.0410 \right) \frac{N_c^2 - 1}{N_c^2} (N_c \sqrt{2 \pi g})^5 T^4 \\
    \gamma &= \frac{N_c g^2 T}{4 \pi} \left(\ln \left(\frac{m_D}{\gamma}\right) + 3.0410 \right) \\
    m_D^2 &= \frac{2N_c + N_f}{6} g^2 T^2,
\end{align}
where $N_c$ the number of "colors" ($N_c = 2$ for the weak SU(2) theory) and $g = g_2$ the coupling constant.

\noindent
In this regime the process is in equilibrium. From considering the
equilibrium relations between chemical potentials for different charges one
can derive the relation between the $B - L$ and the $B$ charges as \cite[sec 2.5]{Leptogenesis_review_doi:10.1146/annurev.nucl.55.090704.151558}
\begin{align}
    \label{eq:sphaleron_conversion}
    B = \frac{8 N_f + 4 N_H}{22 N_f + 13 N_H} (B - L),
\end{align}
where $N_H$ is the number of Higgs doublets ($N_H = 1$ in the SM). We define the prefactor as the sphaleron conversion factor $C_\mathrm{sph}$.
This shows an important requirement for a bayrogenesis model.
A baryogenesis process has to produce a $B - L$ asymmetry otherwise the sphalerons
will washout the asymmetry in $B$ if the process happens in the symmetric (un-broken) phase. Since $B - L$ is a conserved quantity in the standard model, physics beyond the standard model is needed.

\subsection{The Weinberg Operator: Effective Lepton-Number Violations}
\label{sec:lepton_number_violations}

In effective field theory one can generate lepton violations at dimension five only from the
Weinberg operator \cite{Weinberg_operator_original_PhysRevLett.43.1566}. It is given by \cite[eq. 2]{Weinberg_at_colliders_Fuks_2021}
\begin{align}
\mathcal{L}_5 = \frac{C_{l l'}}{\Lambda} [H \cdot \bar{L}^c_l] [L_{l'} \cdot H] + h.c.
\end{align}
where $L = (\nu_l,\, l)^T$ is the left handed lepton doublet ($\nu_l$ the left handed neutrino and $l$ the charged lepton) and $H = (h_+,\, h_0)^T$ the Higgs doublet. The indices $l$ and $l'$ denote the flavor index and the parameter $C_{l l'}$ encodes a potential flavor dependence (which we will ignore).
The superscript $c$ defines charge conjunction, which acts on spinors as $C \psi C = i \gamma^2 \psi^*$
and $\Lambda$ is the energy scale of this operator.
As one can see this operator only contains left-handed leptons and no anti-leptons. Therefore
it generates scattering processes which violate $L$ by $\Delta L = 2$.
At lower energies, below the electro-weak phase-transition where the Higgs field takes a vacuum expectation value, this operators generates a mass-term for left-handed neutrinos of the form
\begin{align}
C_{ll'} \frac{v_{\mathrm{EW}}^2}{\Lambda} C v_l v_l',
\end{align}
where $\langle H \rangle = (0,\, v_{\mathrm{EW}} / \sqrt{2})^T$.
This is a Majorana mass term ????? for the mass matrix $m_{\nu, ll'} = C_{ll'} \frac{v_{\mathrm{EW}}^2}{\Lambda}$
We assume flavor independence and that the sum of neutrino masses
is of order the atmospheric neutrino mass difference $\bar{m}^2 = \sum_i m_{\nu, i} \sim \Delta m_{\mathrm{atm}} = 2.4 \cdot 10^{-3} \, \mathrm{eV}^2$ as in \cite{Kusenko_2015_Axion_Leptogenesis}.
Then the effective cross-section for $\Delta L = 2$ processes
is given by \cite[eq. 9]{Kusenko_2015_Axion_Leptogenesis}
\begin{align}
\label{eq:weinberg_rate}
\sigma_{\Delta L = 2} \approx \frac{3}{32 \pi^2} \frac{\bar{m}^2}{v_{\mathrm{EW}}^4}.
\end{align}
This approximation is valid if the center-of-mass energies in the processes are always small compared to the energy scale $\Lambda$.
There are different possible UV completions that would generate the Weinberg operator
One is the see-saw mechanism which introduces heavy right handed neutrinos and explains the low masses of the left-handed neutrinos.
In this case the Weinberg operator is mediated by right handed neutrinos and
couples to the left-handed lepton and the Higgs doublets via Yukawa couplings.
If one integrates out the right handed neutrino by assuming that their mass is
large compared to the center-of-mass energy (essentially neglecting the momentum in the denominator of the internal propagator), one obtains the Weinberg operator.
For this approximation to be valid in our computations, we require that
the masses of the right-handed neutrinos are large compared to the temperatures
of the thermal plasma. Then the center-of-mass energy of the scatterings are
small compared to the mass of the mediating right-handed neutrinos.
Also this ensures that the right-handed neutrinos are not produced in the plasma and therefore there is not interference with other mechanisms for leptogenesis.

\subsection{Baryogenesis Models}
\label{sec:baryogenesis_models}
In this section we will briefly describe several possible models for baryogenesis but will not go into detail for any of them.
We give a source that serves as a pedagogical introduction for each of them.

\begin{itemize}
    \item The oldest mechanism for baryogenesis is baryogenesis in grand unified theories (GUT-BG).
    For instance in an SU(5) GUT all requirements for successful baryogenesis are
    fulfilled. This theory contains gauge bosons $X^\mu$ and Higgs bosons $Y$ with decays that directly
    violate baryon and/or lepton number. Given appropriate complex phases in the Yukawa couplings
    their decays violate CP. The decay of particles is a non-equilibrium process. Hence the Sakharov conditions are satisfied and indeed successful baryogenesis can be realized as detailed computations show.
    Unfortunately this mechanism requires a reheating temperature that is too high to avoid other constrains. For a detailed discussion see ref. \cite[sec. 3]{Dine_2003_Bayrogenesis}.
    \item Another mechanism for baryogenesis is electro-weak baryogenesis (EW-BG).
    This model implements baryogenesis at the electro-weak phase transition.
    Above symmetry breaking the net baryon number is zero.
    At the phase transition bubbles where the electro-weak symmetry $SU(2)_W \times U(1)_Y$ is broken might form, which then
    move through the plasma.
    Scatterings with the bubble wall create an asymmetry in CP and C in front of the bubble if C and CP violating interactions are present.
    Those asymmetries bias the sphaleron rate to create more particles than anti-particles in the unbroken phase.
    Some particles then are collected by the bubble. Inside the bubble the sphaleron rate is suppressed and the
    baryon number is not washed-out.
    The formation of bubbles only happens if the elector-weak phase-transition is first order.
    Unfortunately this is only the case in the standard model if the mass of the Higgs boson is much lower than
    the measured value.
    This property is modified if additional particles are introduced which then modify the effective finite-temperature
    Higgs-potential that determines whether the phase-transition is first order.
    For instance the minimal supersymmetric standard model (MSSM) would allow for this mechanism to happen.
    One appealing aspect of the theory is that the additional couplings need to appear at energies accessible to future collider
    experiments. An overview over this field is given in \cite{Electroweak_baryogenesis_Morrissey_2012}.
    \item Leptogenesis is a mechanism that requires comparatively little additional assumptions beyond the standard model.
    A natural way to explain the small neutrino masses in the standard model is to introduce heavy right handed neutrinos via the see-saw mechanism.
    Those neutrinos could be produced in the early universe and then decay into leptons violating lepton number as well as C and CP. No particles carrying baryon
    number take part in this process.
    Therefore a $B - L$ asymmetry is generated that is converted into baryons by the sphalerons. An introduction to the model is given in \cite{Leptogenesis_review_doi:10.1146/annurev.nucl.55.090704.151558} and \cite{Leptogenesis_Buchmüller:2014} while more technical details are discussed in \cite{Pedestrians_Buchm_ller_2005}.
    \item The last model we will discuss in this section is Affleck-Dine baryogenesis (AD-BG).
    In this mechanism a scalar field carrying baryon number is introduced (such as in a super-symmetric model).
    This scalar field has a cosmological evolution leading to a high occupation number. This scalar field then decays into standard model particles, creating the baryon asymmetry today.
    This model works at lower energy-scales and it can also explain why the abundance of baryonic and dark matter is so similar. An introduction to AD-BG can be found in \cite[Part III]{Dine_2003_Bayrogenesis}.
\end{itemize}


\section{Spontaneous Leptogenesis with Axions}
\label{sec:spontaneour_bayrogenesis}
This second half of the report will deal with the model that will be
topic of the thesis project namely the implementation of \emph{spontaneous baryogenesis} using \emph{axion}.

\subsection{Spontaneous Baryogenesis}
\label{sec:spontaneour_bayrogenesis_tradiaonal}

All the mechanism above mentioned fulfill the Sakharov conditions.
In refs. \cite{COHEN1987251}, \cite{COHEN1988913} Cohen and Kaplan pointed out that there is a loop hole in the Sakharov conditions.
This loop hole is connected to the assumption that the theory is invariant under CPT.
In general any QFT with a Lorentz invariant groundstate is CPT symmetric but in the early universe the ground state of the theory not Lorentz invariant due to being at finite temperature and more importantly
due to the expanding background. In particular the expectation value of a scalar field and its time derivative will not vanish in general. A coupling to such a field such as \cite[eq. 1.2]{COHEN1987251}
\begin{align}
    \label{eq:sbg_coupling_classical}
    \mathcal{L} \ni - \frac{1}{f} \phi \partial_\mu J^\mu_B
\end{align}
violates CPT. Here $\phi$ is the scalar field, $f$ some energy scale of this new physics and $J_B$ the baryon current.
By partial integration and by assuming that the field $\phi$ is homogeneous one finds \cite[1.3]{COHEN1987251}
\begin{align}
    \frac{1}{f} J_B^\mu \partial_\mu \phi = \frac{1}{f} J_B^0 \dot{\phi}
    %= \frac{1}{f} (n_b - n_{\bar{b}}) \dot{\phi}
    = \frac{1}{f} n_B \dot{\phi}.
\end{align}
If one then compares this term with the density matrix of the grand-canonical ensemble
\begin{align}
    \mathcal{Z} = e^{-\beta H + \beta \mu N}
\end{align}
one finds that the chemical potential $\mu$ is modified by this
interaction as
\begin{align}
    \mu \to \mu_\mathrm{eff} = \mu + \frac{\phi}{f}
\end{align}
If the baryon asymmetry is not protected i.e. there exist B-violating processes in the thermal plasma, the baryon asymmetry is then given in equilibrium as \cite[eq. 3.58]{the_early_universe_kolb_and_turner}
\begin{align}
    n_B^\mathrm{eq} = \mu_\mathrm{eff} \frac{T^2}{6}
\end{align}
In this case the final baryon asymmetry is given by the value of $n^\mathrm{eq}_B$ at the freeze-out of the baryon number violating interaction.
This mechanism indeed does not require C and CP violation because the bias of the interaction towards baryons comes from the chemical potentials and not from the cross-sections of the interactions.
Note that still C and CP violation are allowed to occur.
Furthermore this mechanism also works in equilibrium.
This is the "classic" picture of how spontaneous baryogenesis works. Later we will discuss a more general picture
that also works for more general couplings then eq. \eqref{eq:sbg_coupling_classical}.
This approach will also be needed for the implementation of the model with the axion as the case discussed here is too
specific. Nevertheless this gives an intuitive picture of what the fundamental idea behind SBG is.

\subsection{Axions}
\label{sec:axions}

For our model we will use an axion-like field for the scalar-field.
The axion is a pseudo-scalar field that arises as a Nanbu-Goldstone boson $\phi = f_a \arg \Phi$ of the spontaneous symmetry breaking of some UV completing field $\Phi$ with the axion decay constant $f_a$.
We consider the effective axion Lagrangian that arises as the low-energy description of the axion field \cite[sec. 2.5]{Di_Luzio_2020_Landscape_of_QCD_Axion_models}
\begin{align}
    \mathcal{L}_{\mathrm{axion}} &= \mathcal{L}_{\mathrm{axion-eff}} + \mathcal{L}_{\mathrm{axion-SM}} \\
    \mathcal{L}_{\mathrm{axion-eff}} &= - \frac{1}{2} (\partial_\mu \phi) (\partial^\mu \phi) - V_\mathrm{eff}(\phi; T) \\
    \mathcal{L}_{\mathrm{axion-SM}} &=
        - \frac{g_2^2 C_{W\tilde{W}}}{32 \pi^2} \frac{\phi}{f_a} W_a^{\mu \nu}\tilde{W}^a_{\mu \nu}
        - \frac{g_3^2 C_{G\tilde{G}}}{32 \pi^2} \frac{\phi}{f_a} G_a^{\mu \nu}\tilde{G}^a_{\mu \nu}
        - C_{L} \frac{\partial_\mu \phi}{f_a} J^\mu_L,
\end{align}
where $W$ is the SU(2) field strength tensor of the electro-weak interaction and $G$ the SU(3) field strength tensor for the strong interaction, $\tilde{W}$ and $\tilde{G}$ are their duals, $f_a$ the axion decay constant and $J_L$ the lepton current.
The dimension-less parameters $C_{W\tilde{W}}$, $C_{G\tilde{G}}$ and $C_L$ are taken to be either vanishing or unity in this work.
Note that by rotating the fermion fields one can get rid of either the anomulus term of the form $\sim F \tilde{F}$ or the coupling to the current of the form $\sim \partial \phi J$
or a linear combination of them. This way one could only write two coupling terms or
introduce a coupling to the baryon current ??????. The choice of field basis use above will be useful later.
The axion field forms a condensate in a coherent state that is governed by classical
equations of motion. Hence the cosmological evolution of the axion field is given by
the Klein Gordon equation on an FLRW background \cite[chap. 1, 2]{the_early_universe_kolb_and_turner} with the effective potential $V_\mathrm{eff}$ for the zero-mode i.e. \cite[sec. 4.1, 4.2]{Axion_Cosmology_Marsh_2016}
\begin{align}
    \label{eq:axion_klein_gordon}
    \ddot{\phi} + 3 H \dot{\phi} + \frac{\partial V(\phi; T)}{\partial \phi} = 0.
\end{align}
The effective potential for the axion field is generated from instanton effects.
For QCD axion from SU(3) instanton in the standard model strong sector and hidden sector axions from some hidden strongly coupled sector. It is then given approximately as \cite[sec. 2.2]{Axion_Cosmology_Marsh_2016}
\begin{align}
   &V_\mathrm{eff}(\phi; T) = m_a^2(T) (1 - \cos(\phi / N f_a)) \\
   &m_a^2(T) = m_a(T = 0) \cdot \min\left(1, \left( T / \Lambda \right)^p \right)
\end{align}
in the dilute instanton gas / one-instanton approximation.
The integer value $N$ is the axion domain-wall number, which we set to $N = 1$ as in the DFSZ-model \cite[sec. 2.7]{Di_Luzio_2020_Landscape_of_QCD_Axion_models}.
The form above should be used for a detailed analysis. But for simplicity we assume a temperature independent harmonic potential
\begin{align}
    V_\mathrm{eff} = \frac{1}{2} m_a^2 \phi^2,
\end{align}
where the axion mass $m_a$ is the only free parameter.
The cosmological evolution of the axion field is as follows. We assume that the
symmetry related to the axion UV completion is broken before or during inflation.
Then the axion field is homogeneous over one Hubble patch and is described by eq. \eqref{eq:axion_klein_gordon} \cite[sec. 4, eps. 4.7]{Axion_Cosmology_Marsh_2016}.
If the potential and kinetic energy are of the same order, any initial motion of the axion is damped by Hubble friction. Therefore one can assume the initial conditions $\phi = \phi_i$ and $\dot{\phi} = 0$ at the end of inflation \cite[sec. IV.A]{Axion_cosmology_Wantz_2010}.
Once the Hubble parameter is succinctly low $3 H(t_\mathrm{osc}) \sim m_a(T(t_\mathrm{osc}))$ for the axion to overcome Hubble friction the axion rolls down its potential and starts to oscillate. This roll-down generates the time derivative
required for the baryogenesis process.
The coherent oscillations have a matter-type equation of state and therefore behave as non-relativistic matter. This makes axions a dark matter candidate as long as the axion
doesn't decay.
This decay might happen from coupling to the electro-weak SU(2) field and happens at a decay rate of \cite[eq. 15]{Axion_leptogenesis_Kusenko_2015}
\begin{align}
    \Gamma_a = \frac{g_2^2}{256 \pi^4} \frac{m_a^3}{f_a^2}.
\end{align}

\subsection{Reheating}
\label{sec:reheating}
The violation of lepton number comes from the Weinberg operator (discussed in section \ref{sec:lepton_number_violations}).
The temperature scale for our model is roughly given by the equilibration temperature of the Weinberg operator i.e.
\begin{align}
    \Gamma_W(T_\mathrm{eq}) = H(T_\mathrm{eq}) \Rightarrow T \approx 10^{13} \, \mathrm{GeV}
\end{align}
This means this process has to happen during/directly after reheating. The reheating epoch happens right after inflation. The inflaton field, that has driven by exponential expansion of the universe during inflation has started to oscillate and decays into standard model radiation  \cite[sec. 8.3]{the_early_universe_kolb_and_turner}.
The reheating process is described by the energy rate equations and the Friedmann equation \cite[eq. 8.29, 8.31]{the_early_universe_kolb_and_turner}
\begin{align}
    \label{eq:reheating_equations}
    \dot{\rho_\phi} + 3 H \rho_\phi &= - \Gamma_\phi \rho_\phi \\
    \dot{\rho_R} + 4 H \rho_R &= + \Gamma_\phi \rho_\phi \\
    3 H^2 M_\mathrm{pl}^2 &= \rho_\mathrm{tot} \approx \rho_\phi + \rho_R,
\end{align}
where $\rho_\phi$ is the energy density of the inflaton field, $\rho_R$ the radiation energy density and $\Gamma_\phi$ the effective decay rate of the inflaton to SM particles.
We use the instantaneous thermalization approximation for the temperature
\begin{align}
    \rho_R = \frac{\pi^2}{30} g_{*}(T) T^4,
\end{align}
i.e. we assume that after the decay of an inflaton particle, the decay products immediately follow a thermal distribution, which is not true exactly.
The temperature start at zero since no radiation exits. Then is quickly, within one dynamical time, rises to a maximal
value \cite[eq. 8.33]{the_early_universe_kolb_and_turner}
\begin{align}
    T_\mathrm{max} = 0.8 g_*{-1/4} \rho_0^{1/2} (\Gamma_\phi 8 \pi M_\mathrm{pl})^{1/4},
\end{align}
where $\rho_0 = 3 M_pl^2 H_\mathrm{inf}^2$ is the initial energy density.
After that the energy density is still dominated by the inflaton field (matter domination) until
$\rho_\phi = \rho_R$, were the reheating temperature \cite[eq. 8.34]{the_early_universe_kolb_and_turner}
\begin{align}
    T_\mathrm{RH} = 0.55 g_*^{-1/4} (8 \pi M_\mathrm{pl} \Gamma_\phi)^{1/2}
\end{align}
is reached. At this point the reheating epoch is over and the universe continues to evolve radiation dominated.



\subsection{Implementation of the Model: Problem of Field Basis and the Transport Equation}
\label{sec:model}

Let us now implement a spontaneous baryogenesis model with an axion as the scalar field.
Its evolution is governed by eq. \eqref{eq:axion_klein_gordon}.
Let us first start with a model with $C_L = 1$ and $C_{W \tilde{W}} = C_{G \tilde{G}} = 0$.
In this case the "classic" model from section \ref{sec:spontaneour_bayrogenesis_tradiaonal} applies.
We need to track the asymmetry of the leptons and do so using the Boltzmann equation \cite[eq. 7]{Axion_leptogenesis_Kusenko_2015}
\begin{align}
    \label{eq:L_violation_Boltzmann}
    \dot{n}_L + 3 H n_L &= - \Gamma_W (n_L - n_L^\mathrm{eq}) \\
    \Gamma_W &= n^\mathrm{eq}_L \sigma_W,
\end{align}
where $n_L$ is the number density of leptons vs anti-leptons, $n_L^\mathrm{eq} = \mu_\mathrm{eff} \frac{T^2}{6}$ its equilibrium value and $\sigma_W$ the effective cross-section for the lepton number
changing interactions. For this cross-section we assume the value for Weinberg operator from the see-saw mechanism as discussed in section \ref{sec:lepton_number_violations}.
The final value of the asymmetry is determined by integrating the Boltzmann eq. \eqref{eq:L_violation_Boltzmann}, the Klein-Gordon eq. \eqref{eq:axion_klein_gordon} and the
equations for reheating \eqref{eq:reheating_equations} numerically.
The value of $\eta_L = n_L / n_\gamma$ will converge to
a constant value after the rate of the lepton number violating interaction is sufficiently slow compared to the Hubble rate.
We also include the dilution by the decay of the axion. This computation is based on the same equations
as the reheating process (eq. \eqref{eq:reheating_equations}) with the axion field instead of the
inflaton field. The initial conditions are set from the end of the simulation of the baryogenesis
processs.
After evaluating $\eta_L$ after the axion decay at $t \sim 1/\Gamma_a$, it can then be translated to
the present day baryon asymmetry using the sphaleron conversion factor in eq. \eqref{eq:sphaleron_conversion} and the redshift to today from eq. \eqref{eq:asymmetry_redshift}.
The numerical result for this model is shown in figure \ref{fig:naive_model_plot}.
\begin{figure}
    \centering
    \includegraphics[width=0.8\linewidth]{../../plots/evolution.pdf} % evolution of the asymmetry plot
    \includegraphics[width=0.8\linewidth]{./../../plots/transport_vs_boltzmann.pdf} % parameter space plot
    \caption{"Preview" plots for the simulation of the spontaneous baryogenesis process with axions and via leptogenesis.
    \textbf{Top:} The x-axis is time in GeV on logarithmic axis. A lot of an example evolution for specific parameters. On the top subplot the evolution of the temperature in GeV is shown on a log scale. Equilibration times for interactions that are within the temperature range are shown as dashed vertical lines. In the lower subplot the evolution of the chemical potential of $B - L$ over the temperature is shown with a logarithmic axis on the left. The line computed with full reheating is black while result in radiation domaination only is shown in red. The source from the axion $\dot{\theta} / T$ is shown as a blue line with a linear scale on the right.  \textbf{Bottom:} The logarithm of the resulting asymmetry as a function of the inflaton decay rate $\Gamma_\phi$ in GeV on a log-scale on the y-axis and the axion mass in GeV on a log-scale on the x-axis with fixed decay constant $f_a$ and Hubble parameter at the end of inflation. The result computed with the transport eq. is shown as solid lines and the result computed with the Boltzmann eq. as dashed lines. The observed baryon asymmetry is indicated by a red line for both cases. }
    \label{fig:naive_model_plot}
\end{figure}¸
We will not discuss the result in detail here and postpone these discussion to the thesis itself.
Instead we will now discuss an issue related to the computation of the asymmetry and the coupling of the axion.

The model we discussed so far is based on ref. \cite{Axion_leptogenesis_Kusenko_2015}.
There the authors argued that the coupling originates from an axion coupling to $W \tilde{W}$ ($C_{W \tilde{W}} = 1, C_L = C_{G \tilde{G}} = 0$) and then replacing the $W \tilde{W}$ term using
the anomolus equation (eq. \eqref{eq:B_and_L_current_equations}).
The problem with this approach is that in the presents of the Weinberg operator eq. \eqref{eq:B_and_L_current_equations} is not true but contains in extra term from the Weinberg operator as
is also violates L. Therefore rewriting the coupling this way would also introduce a coupling of
the axion to the Weinberg operator, making the rate in eq. \eqref{eq:L_violation_Boltzmann} incorrect.
Instead one has to do a field redefinition of the form (here for some fermion fields $l_n$)
\begin{align}
    l_n \to e^{i c_n \phi / f_a} l_n
\end{align}
to remove the anomalous coupling. This would introduce a axion dependent phase in the Weinberg operator
but this does not effect the Born-level cross-section in eq. \eqref{eq:weinberg_rate}.
Unfortunately as pointed out in ref. \cite{Shi_2015_Basis_Invariance_chemical_equilibrium} this induces an energy shift from the kinetic term \cite[eq. 6]{Shi_2015_Basis_Invariance_chemical_equilibrium}
\begin{align}
    - c_n \dot{\phi} / {f_a} l_n^{\dagger} l_n.
\end{align}
The energy shift cancels the energy shift from the coupling to the lepton current, making the effective
chemical potential invariant under changes of the field basis i.e. there is no simple basis in which
one can simply read of its value.
This means that such a coupling can not be interpreted just as a shift of chemical potential.
The coupling still biases the reactions.
As an alternative the authors of ref. \cite{Domcke:2020kcp_Generic_Couplings} developed a formalism
to avoid this issue and that is intrinsically field basis independent.
The equation of motion is here the transport equation for all standard model charges $q_i$.
It is
given as \cite[eq. 3.4]{Domcke:2020kcp_Generic_Couplings}
\begin{align}
\label{eq:transport_equation}
\dot{q}_i + 3 H q_i = - \sum_\alpha \Gamma_\alpha n^\alpha_i \Big[
\sum_j n^\alpha_j \left( \frac{\mu_j}{T} \right) - n_S^\alpha \frac{\dot{a} / f_a}{T} \Big].
\end{align}
This equation was obtained via linear response theory considering the operator equation \cite[eq. 2.1]{Domcke:2020kcp_Generic_Couplings}
\begin{align}
    \partial_\mu J^\mu_i = \sum_\alpha n^\alpha_i \mathcal{O}_\alpha,
\end{align}
where $J_i$ is the current for the charge $i$, $\mathcal{O}_\alpha$ some interaction operator and
$n^\alpha_i$ encodes the change of charge $i$ in the process $\alpha$.
The relevant charges for our case are \cite[eq. 5.6]{Domcke:2020kcp_Generic_Couplings}
\begin{align}
i \in (\tau, L_{12}, L_3, q_{12}, t, b, Q_{12}, Q_3, H),
\end{align}
where the number indicate fermion generations, $L/Q$ the left-handed lepton/quark doublet, $q$ the right handed quark, $t,b$ the right handed top/bottom, $\tau$ the right handed tau and $H$ the Higgs.
The rates in eq. \eqref{eq:transport_equation} are formally defined as \cite[eq. B.18]{Domcke:2020kcp_Generic_Couplings}
\begin{align}
    &\Gamma_\alpha = - \frac{T G_{\alpha \alpha}^\rho(\omega, \mathbf{0})}{2\omega} \Big|_{\omega = 0},
\end{align}
where $G_{\alpha \beta}^\rho(\omega, \mathbf{p})$ is the Fourier transform of the Green function \cite[eq. B.16]{Domcke:2020kcp_Generic_Couplings}
\begin{align}
    G^\rho_{\alpha, \beta}(x - x') = \langle [\mathcal{O}_\alpha(x), \mathcal{O}_\beta(x')] \rangle_C,
\end{align}
where the $C$ subscript means that the expectation value is to be taken in the canonical ensemble.
Therefore these rates have to be computed in thermal field theory.
The relevant processes include the sphaleron transitions discussed in section \ref{sec:sphaelrons},
the Weinberg operator as discussed in section \ref{sec:lepton_number_violations} that violates $B - L$ and hence allows for baryogenesis to happen as well as
Yukawa couplings (weak sphaleron, strong sphaleron, top Yukawa, top Yukawa, bottom Yukawa, Weinberg for generations 1 and 2, Weinberg for generation 3).
These are discussed in \cite[sec. 3.1]{Domcke:2020kcp_Generic_Couplings} and the corresponding charge vectors are given in \cite[eq. 5.7]{Domcke:2020kcp_Generic_Couplings}.
The source term from the axion is encoded in the vector $n_S^\alpha$ describing the couplings of
the axion to the interactions $\mathcal{O}_\alpha$. It is $n_S^\alpha = \delta_{\alpha \beta}$ for a coupling of the form $\frac{\phi}{f_a} \mathcal{O}_\beta$
and $n_S^\alpha = n_k^\alpha$ for a coupling of the form $\partial_\mu \frac{\phi}{f_a} J^\mu_k$.
For a coupling to $W \tilde{W}$ it is given as $n_S^\alpha = (1, 0, 0, 0, 0, 0, 0)$.
The important property of this transport equation compared to a naive Ansatz with a
Boltzmann equation is that it is independent of the field basis.
Finally let us look at the intuition for spontaneous baryogenesis in this more field theoretic rather then kinetic picture (adapted from \cite[sec. 1]{Domcke:2020kcp_Generic_Couplings}).
We consider a toy theory with two charges $Q_1$ and $Q_2$. The baryon charge is given as the
linear combination $B = Q_1 - Q_2$. There are two interaction $\mathcal{O}_X$ and $\mathcal{O}_Y$ with the operator current equations
\begin{align}
    \partial_\mu J_1^\mu &= \mathcal{O}_X \\
    \partial_\mu J_2^\mu &= \mathcal{O}_Y - \mathcal{O}_X
\end{align}
Hence the equation for the baryon number is
\begin{align}
    \partial_\mu J_B^\mu = \mathcal{O}_Y.
\end{align}
The picture we discussed above works by introducing a coupling $\phi / f_A \mathcal{O}_Y$,
which biases the equilibrium B charge away from zero.
But this also works in a more general setting.
Coupling the scalar field to $\mathcal{O}_X$ will bias the $Q_1$ charge away from zero.
At the same time the interaction $\mathcal{O}_Y$ will washout $Q_2$ ie. drive the $Q_2$ charge towards zero. In total this means that $B$ will obtain a non-vanishing value. Once $\mathcal{O}_Y$ freezes out i.e. its rate will be low compared to the Hubble rate
the B abundance is fixed at a finite value.
This happens generically in the scenario we are considering. The standard model processes
play the role of $\mathcal{O}_Y$ and are active also at lower temperatures while the Weinberg operator plays the role of $\mathcal{O}_X$ and freezes out at higher temperatures.
The coupling of the axion can be anything unless it couples to a charge vector that is orthogonal to the desired B charge vector e.g. to the charge $- Q_1 + Q_2$ with the coupling $\phi / f_a (- \partial_\mu J_1^\mu + \partial_\mu J_2^\mu) = \phi / f_a (- 2 \mathcal{O}_X + \mathcal{O}_Y)$.
This indeed allows for the possibility to generate the lepton charge with the coupling to the e.g.
strong sphaleron $G \tilde{G}$ as QCD axions have.

\newpage
\subsection{Outlook}
\label{sec:outlook}

In the following master's thesis project we will compute the baryon asymmetry using the transport equation \eqref{eq:transport_equation} during reheating (sec. \ref{sec:reheating}) and
solving the Klein Gordon equation for the axion field (eq. \eqref{eq:axion_klein_gordon}).
This combines the results of ref. \cite{Axion_leptogenesis_Kusenko_2015} in which the reheating process and the motion of the axion are correctly taken care of but the asymmetry production
is not correctly computed (see sec. \ref{sec:model}) and those of
ref. \cite{Domcke:2020kcp_Generic_Couplings} which developed the formalism to correctly compute
the asymmetry while not considering the reheating process and the motion of the
axion \footnote{In this paper they considered the process only in the radiation dominated epoch. While their results are correct in this respect, they can be extended to larger parts of the parameter-space by considering the reheating process.}.
Furthermore we will analyze the parameter space in the spirit of ref. \cite{Axion_leptogenesis_Kusenko_2015} (see sec. \ref{sec:model}) to obtain bounds
on the parameters for successful baryogenesis.

Building on this, we can modify the model in several ways.
First we could change the nature and scale of the lepton violating interactions.
Then we can change the coupling of the axion as discussed above to e.g. the strong sphaleron.
Finally we could change the dynamics of the axion. This was for instance done in ref. \cite{Deformed_potential_Bae_2019}.
In this work the dynamics of a field depended wave-function renormalization was considered,
where the kinetic term of the axion is multiplied by a function that depends on the value of
the axion field, leading to a deformed potential for a canonical field defined in terms of
the axion.
This model allows a non-vanishing velocity of the axion at high energy scales as required for
SBG while also having a low axion mass at late times making the axion stable.
It allows the axion to explain baryogenesis and the dark matter at the same time.
We are planing to apply the transport equation method to this model to obtain correct results
for the asymmetry. A particular implementation of this idea was shown in \cite[sec III]{Deformed_potential_Bae_2019} to be the continuous clockwork axion.
In this model a moduli parameter $mR$ appears that controls the compactification from which the
axion arises. In ref. \cite{Deformed_potential_Bae_2019} this parameter is
set to specific values (see \cite[fig. 2]{Deformed_potential_Bae_2019}).
A more complete model has to consider moduli-stabilization where the value of $mR$ is determined dynamically. This will also be an area of interested in the coming work.
Finally we might be interested in the dynamics of multiple scalar fields.
The interesting aspect here is that the motion of the axion field that drives baryogenesis
is not mainly controlled by its mass but by the coupling to the other field. In some parts of
the parameter space (including relevant ones for BG) the dynamics seems to be chaoic, making
the analysis more complicated.


\section{Conclusion}
\label{sec:conclusion}
In this report we have discussed the basics of baryogenesis.
This included the discussion of sphalerons and the lowest-order effective lepton number violating term the Weinberg operator.
Building on this we have considered a lesser known model for baryogenesis, spontaneous baryogenesis. This model can be implemented using an axion-like field during or shortly after reheating with the Weinberg operator as the source of lepton number violations.
We elaborated on an intuition on how the mechanism works as well as a subtlety related to the computation of the asymmetry.
Let us conclude by stating that we believe that
this research area is promising as it is comparatively little explored while a lot of different scenarios seem possible.


\newpage
\printbibliography
\end{document}

