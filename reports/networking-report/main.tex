\documentclass[a4paper]{article}
%\documentclass[13pt,a4paper,twoside,titlepage]{article}
\usepackage[ngerman, english]{babel}
\usepackage[utf8]{inputenc}
\usepackage{latexsym,amssymb,amsmath,amstext}
\usepackage{graphicx}
\usepackage{float}
\usepackage[backend=bibtex, sorting=none]{biblatex}
\usepackage{amsmath}
\usepackage{ dsfont }
\usepackage{csquotes}
\usepackage{slashed}
\usepackage{subfigure}
\usepackage{listings}
\usepackage{subfigure}
\usepackage{verbatim}
\usepackage{xcolor}
\usepackage{multicol}
%\usepackage{feynmf}
\usepackage{abstract}
\usepackage{bbold}
%\usepackage[a4paper, left=2cm, right=2cm, top=2cm, bottom=3cm, bindingoffset=10mm]{geometry}
\usepackage[compat=1.1.0]{tikz-feynman}
%\usepackage[onehalfspacing]{setspace}
\usepackage{tikz}
\usepackage{pgfplots}
\lstset{
  breaklines=true,
  postbreak=\mbox{\textcolor{red}{$\hookrightarrow$}\space},
}

\newcommand{\Tr}{\operatorname{Tr}}
\newcommand{\note}[1]{{\color{red}[\textbf{#1}]}}
\newcommand{\diff}{\ensuremath{\mathrm{d}}}

\bibliography{refs}

\begin{document}

\section{Introduction}
In this brief report we will report on the patras workshop 2021.
First we describe the scope of the workshop, then two talks on a specific topic.
Finally we describe the online-format and conclude by reviewing the workshop from our own perspective.

\section{The Patras-Workshop}
The 16th Patras workshop on axions, WIMPs and WISPs has been held online from the 14.6.2021 to the 18.6.2021 and was organized by the university of Trieste.
Its main focus was on axions especially axion detection experiments.
Other topics included dark photons dark matter modeling and
detection,  axion stars,  bounds on different dark matter candidates (mostly axions).

\section{Talks on Axion-Strings}
Axion strings are one dimensional topological defects in the UV completing
(PQ) field of the axion. The field winds the field space on a loop
in real space. Contracting the loop to a single point, while this ``winding number'' stays the same yields a point, where the axion field is undefined. In three dimensions this can happen along a 1 dimensional curve - the axion string.
During the cosmological evolution the energy of those strings is transferred into axions.
Hence the axion dark matter relic density is influenced by this process.
The qualitative amount (significant or sub-leading) contribution was well as the precise quantitative amount
of axion produced by cosmic strings is topic of debate for the last decades \cite{Gorghetto_2018}. \\


\noindent
There were two talks given on axion strings.
The first was a short talk by Asier Lopez-Eiguren on ``Approach to scaling in axion string networks'' based on their paper with the same title \cite{Hindmarsh_2021}.
They performed numerical simulations of the string network and compared it to an effective
model. As the result they found that the dynamics is well described by the equations of the effective theory. That means that the `logarithmic violating'' observed by other groups is only a transient effect.
Their result is not fully conclusive as they couldn't simulate the string network to the point, where
the transient effect vanishes. \\

\noindent
The second talk was given by Marco Gorghetto on gravitational wave signals
from axions strings. They found that the signal could be described by an effective theory
and found new bounds on the axion parameter space from gravitational wave detectors \cite{Gorghetto_2021}.

\section{The Online-Conference Format}
Let us now discuss the format of the conference.
Due to the ongoing COVID-19 pandemic the conference was held online using Zoom.
There were 25 minute talks as well as 5 minute presentations that replaced poster presentations
The presentation were given with almost no breaking in between and little room for questions (only one or two
questions here allowed). This lead to the conference always going over time.

\section{Review}
I will comment on the online format from my perspective.
The positive side of the online format is the easier and cheaper (in this case completely free)
attendance, because of lesser or non attendance fees and no travel and accommodation costs.
This makes it easier in particular for students to attend.
The flip side of this is that the barrier for real networking is much higher.
All questions are asked in front of the hole audience. This includes the
poster presentations. The breakout rooms for the break were only used by a few people who apparently already knew
each other.
All this is a bit intimidating. A bout almost all
talks, as a student, I don't know a lot about the topic and the talk is quite advanced.
This is of course always the case, online or not. But here only opportunity is given to have
conversations in smaller groups, which I imagine to be more beginner friendly.
I the end it was interesting to hear about various current research topics but
I didn't really do any networking.


\newpage
\printbibliography
\end{document}

