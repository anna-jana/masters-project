\NeedsTeXFormat{LaTeX2e}[2005/12/01]
%%    2009/03/12 v1.0 GAUBM Vorlage f�r Aschlussarbeiten Physik
%% Template fuer Bachelor- und Masterarbeiten
%% an der Fakultaet fuer Physik (c) Thomas Pruschke der GA Universit�t
%% Verbesserungsvorschlaege bitte an studiendekanat@physik.uni-goettingen.de
%%
%% Benoetigte Pakete: datenumber
%%

%%%%%%%%%%%%%%%%%%%%%%%%%%%%%%%%%%%%%%%%%%%%%%%%%%%%%%%%%%%%%%%%%%%%%%
%%%%%%%%%% Bitte vor dem Veraendern diese Datei umbenennen! %%%%%%%%%%
%%%%%%%%%%%%%%%%%%%%%%%%%%%%%%%%%%%%%%%%%%%%%%%%%%%%%%%%%%%%%%%%%%%%%%

%% scrbook - Ersatz f�r LaTeX book Klasse aus dem KOMA Script
%% Moegliche Optionen: diejenigen der Klasse scrbook ausser titlepage

%% deutsche Arbeit:
\documentclass[master,       %% Typ der Arbeit: bachelor oder master
               twoside,        %% zweiseitiges Layout
               BCOR10mm,       %% Bindekorrektur 10 mm
%               liststotoc,nomtotoc,bibtotoc, %% Aufnahme der div. Verzeichnisse
                                              %% ins Inhaltsverzeichnis
               english,ngerman, %% Alternativspr. Englisch, Dokumentspr. Deutsch
%               ngerman,english  %% Alternativspr. Deutsch, Dokumentspr. Englisch
%               final,          %% Endversion; draft fuer schnelles Kompilieren
               ]{GAUBM}

\usepackage{setspace}  %% Zur Setzung des Zeilenabstandes
\usepackage{babel}     %% Sprachen-Unterstuetzung
\usepackage{calc}      %% ermoeglicht Rechnen mit Laengen und Zaehlern
\usepackage[T1]{fontenc}       %% Unterstutzung von Umlauten etc.
\usepackage[latin1]{inputenc}  %%
%% in aktuellem Linux & MacOS X wird standardmaessig UTF8 kodiert!
%\usepackage[utf8]{inputenc}    %% Wenn latin1 nicht geht ...

\usepackage{amsmath,amssymb} %% zusaetzliche Mathe-Symbole

\usepackage{lmodern} %% type1-taugliche CM-Schrift als Variante zur
                     %% "normalen" EC-Schrift
%% Paket fuer bibtex-Datenbanken
\usepackage[comma,numbers,sort&compress]{natbib}
\bibliographystyle{plainnat}

\newcommand{\tabheadfont}[1]{\textbf{#1}} %% Tabellenkopf in Fett
\usepackage{booktabs}                      %% Befehle fuer besseres Tabellenlayout
\usepackage{longtable}                     %% umbrechbare Tabellen
\usepackage{array}                         %% zusaetzliche Spaltenoptionen

%% umfangreiche Pakete fuer Symbole wie \micro, \ohm, \degree, \celsius etc.
\usepackage{textcomp,gensymb}

%\usepackage{SIunits} %% Korrektes Setzen von Einheiten
\usepackage{units}   %% Variante fuer Einheiten

%% Hyperlinks im Dokument; muss als eines der letzten Pakete geladen werden
\usepackage[pdfstartview=FitH,      % Oeffnen mit fit width
            breaklinks=true,        % Umbrueche in Links, nur bei pdflatex default
            bookmarksopen=true,     % aufgeklappte Bookmarks
            bookmarksnumbered=true  % Kapitelnummerierung in bookmarks
            ]{hyperref}

%% Weiter benoetigte Pakete: datenumber
%% Falls dieses Paket nicht in der Installation vorhanden ist,
%% kann es von der Seite mit diesem Template heruntergeladen werden
%% und in einem LaTeX bekanntem Verzeichnis installiert werden (notfalls
%% dem Verzeichnis mit der Arbeit).
\begin{document}
%%
%%                   Ab hier muessen die Anpassungen geschehen
%%
%% Hier den eigenen Namen einsetzen
\ThesisAuthor{Anna-Jana}{Rie�}
%% Hier den Geburtsort einsetzen
\PlaceOfBirth{Hamburg}
%% Titel Arbeit. Das erste Argument ist der deutsche, das zweite der
%% englische Titel.
\ThesisTitle{Spontane Axion-Leptogenese mit Generischen Kopplungen in ALP Realignment, Clockwork Axion und Multi Axion Modellen}{Spontaneous Axion-Leptogenesis with Generic Couplings in ALP Realignment, Clockwork Axion and Multi Axion Models}
%% Erst- und Zweitgutacher/in
%% Ist der/die Betreuer/in nicht identisch mit dem/r Erstgutachter/in,
%% muss diese/r als optionales Argument angegeben werden.
\FirstReferee{Prof.\ Dr.\ Laura Covi}
\Institute{Institut f\"ur Theoretische Physik}
\SecondReferee{Prof.\ Dr.\ David J. E. Marsh}
%% Beginn und Ende des Anfertigungszeitraumes
\ThesisBegin{1}{2}{2021}
\ThesisEnd{15}{8}{2022}
%% DO NOT TOUCH THESE LINES!!!!
\frontmatter
\maketitle
\cleardoublepage
%% So laesst sich in die andere Sprache umschalten (Englisch bzw. Deutsch)
\begin{otherlanguage}{english}
\begin{abstract}
  Here the key results of the thesis may be presented in about
  half a page.
  \bigskip\par
  \noindent \textbf{Keywords:} Physics, Masters Thesis, Cosmology, The Early Universe, Reheating, Axions, Baryogenesis, Spontaneous Baryogenesis, Leptogenesis, Realignment, Clockwork Theory, String Theory Axions, Non-Linear Dynamics, Thermal Field Theory, Linear Response, Transport Equation, Numerical Simulation
\end{abstract}

%% Ende des Vorspanns
\cleardoublepage
%% Ab hier 1 1/2 facher Zeilenabstand (durch setspace-Paket)
\onehalfspacing
%% Erzeugt Inhaltsverzeichnis
\tableofcontents

%%% Hier kann man seine Bezeichnungsweisen erklaeren. Falls nicht
%%% benoetigt, bis einschliesslich \end{nomenclature} auskommentieren
%\begin{nomenclature}
%%% Fuer die Berechnung der Spaltenbreiten muss \usepackage{calc}
%%% geladen sein!
%\section*{Lateinische Buchstaben}
%\noindent
%\begin{longtable}[l]{p{0.2\textwidth}p{0.7\textwidth-6\tabcolsep}p{0.1\textwidth}}
%  \tabheadfont{Variable}&\tabheadfont{Bedeutung}&\tabheadfont{Einheit}\\\midrule\endhead
%  $A$ & Querschnittsfl"ache & $\unit{m^2}$\\
%  $c$ & Geschwindigkeit & $\unitfrac{m}{s}$
%\end{longtable}
%\section*{Griechische Buchstaben}
%\begin{longtable}[l]{p{0.2\textwidth}p{0.7\textwidth-6\tabcolsep}p{0.1\textwidth}}
%  \tabheadfont{Variable}&\tabheadfont{Bedeutung}&\tabheadfont{Einheit}\\\midrule\endhead
%  $\alpha$  & Winkel & $\unit{\degree}$; --\\
%  $\varrho$ & Dichte & $\unitfrac{kg}{m^3}$
%\end{longtable}
%\section*{Indizes}
%\begin{longtable}[l]{p{0.2\textwidth}p{0.8\textwidth-4\tabcolsep}}
%  \tabheadfont{Index}&\tabheadfont{Bedeutung}\\\midrule\endhead
%  m & Meridian\\
%  $r$ & Radial
%\end{longtable}
%\section*{Abk"urzungen}
%\begin{longtable}[l]{p{0.2\textwidth}p{0.8\textwidth-4\tabcolsep}}
%  \tabheadfont{Abk"urzung}&\tabheadfont{Bedeutung}\\\midrule\endhead
%  2D & zweidimensional\\
%  3D & dreidimensional\\
%  max & maximal
%\end{longtable}
%\end{nomenclature}
%% \listoftables und \listoffigures sollten nur bei genuegender Anzahl Tabellen
%% verwendet werden
%\listoffigures
%\listoftables

\mainmatter   %% Anfang Hauptteil

\chapter{Introduction}

\chapter{Fundamentals}

\section{Early Universe Cosmology}

\begin{itemize}
	\item FLRW metric
	\item Friedmann equations
	\item power law solutions
	\item equilibrium thermodynamics with $g_*$
	\item boltzmann equations
	\item scalar fields
	\item inflation and reheating
\end{itemize}

\section{Baryogenesis}
\begin{itemize}
	\item asymmetry parameter
	\item dings da conditions
	\item B - L in the SM
	\item quik overview of mechanism 
\end{itemize}

\section{Axions}
\begin{itemize}
	\item $\theta$ parameter
	\item PQ symmetry
	\item goldstone boson 
\end{itemize}

\chapter{Spontaneous Baryogenesis}

\section{Mechanism of Spontaneous Baryogenesis}

\begin{equation}
	\mu \to \mu_\mathrm{eff} = \mu + \frac{\dot{\phi}}{f}
\end{equation}

\section{Transport Equation}

\begin{equation}
	\partial_\mu J_i^\mu = \sum_\alpha n_i^\alpha O_\alpha
\end{equation}

\begin{equation}
	\partial_t \langle J_i^0 \rangle_{\mathrm{GC}} = \sum_\alpha n_i^\alpha \langle O_\alpha \rangle_{\mathrm{GC}}
\end{equation}

\begin{equation}
	\rho_\mathrm{GC} = \frac{e^{- (H - \sum_i \mu_i Q_i) / T}}{\operatorname{Tr} \left[ e^{- (H - \sum_i \mu_i Q_i) / T} \right] }
\end{equation}

\begin{align}
	\mathcal{H}(t, \mathbf{x}) &= \varepsilon \int_{- \infty}^t \, \mathrm{d} t' e^{\varepsilon (t' - t)} \sum_i \mu_i J_i^0(t', \mathbf{x}) \\
	&= \sum_i \mu_i J_i^0(t, \mathbf{x}) - \int_{- \infty}^t \, \mathbf{d} t' e^{\varepsilon (t' - t)} \sum_i \mu_i \sum_\alpha n_i^\alpha O_\alpha(t' , \mathbf{x})
\end{align}

\begin{align}
	\rho \sim e^{- (H - \int \mathrm{d}^3 x \mathcal{H}) / T} = e^{-(H - \sum_i \mu_i Q_i) / T + X}
\end{align}

\begin{equation}
	X = - \sum_{i, \alpha} \frac{\mu_i}{T} n_i^\alpha \int_{- \infty}^t \, \mathrm{d}^4 x' e^{\varepsilon(t' - t)} O_\alpha(x')
\end{equation}

\begin{equation}
	\rho \approx \left[ 1 + T \int_0^{1/T} \, \mathrm{d} \tau e^{-(H - \sum_i \mu_i Q_i) \tau} X e^{(H - \sum_i \mu_i Q_i) \tau} - \langle X \rangle_\mathrm{GC}  \right] \rho_\mathrm{GC} 
\end{equation}

\begin{equation}
	\langle Q_\alpha \rangle = 0
\end{equation}

\section{Properties and Analytical Results}

\chapter{Motion of the Axion Field}

\section{ALP Realignment}

\begin{equation}
	\mathcal{L} = \frac{1}{2} \left( \partial_\mu \phi \right)^2 - \frac{1}{2} m_a^2 \phi^2
\end{equation}

\begin{figure}[h]
    \includegraphics[width=\linewidth]{../../transport_eq_in_time/plots/alp_realignment_field_evolution.pdf}
    \caption{}
\end{figure}

\section{Clockwork Theory Axions and Deformed Potentials}
\begin{figure}[h]
    \includegraphics[width=\linewidth]{../../transport_eq_in_time/plots/cw_potential.pdf}
    \caption{}
\end{figure}

\begin{figure}[h]
    \includegraphics[width=\linewidth]{../../transport_eq_in_time/plots/cw_field_evolution.pdf}
    \caption{}
\end{figure}


\section{Multi Axion Systems}

\begin{equation}
	\mathcal{L} = \frac{1}{2} \sum_i \left( \partial_\mu \phi_i \right)^2 - \sum_i \Lambda_i^4 \left[ 1 - \cos \left( \sum_j Q_{i,j} \, \phi_j / f_j \right) \right] 
\end{equation}

\begin{equation}
	\mathcal{L} = \frac{1}{2} \sum_i \left( \partial_\mu \phi_i \right)^2 - \frac{1}{2} \sum_i m_i^2 \phi_i^2 - \sum_{i, j}
	g_{i, j} \phi_i^2 \phi_j^2
\end{equation}

\begin{figure}[h]
    \includegraphics[width=\linewidth]{../../transport_eq_in_time/plots/multi_axion_canonical_field_evolution.pdf}
    \caption{}
\end{figure}

\begin{figure}[h]
    \includegraphics[width=\linewidth]{../../transport_eq_in_time/plots/multi_axion_mass_eigenstate_field_evolution.pdf}
    \caption{}
\end{figure}

\begin{figure}[h]
    \includegraphics[width=0.5\linewidth]{../../transport_eq_in_time/plots/multi_mass_eigenstate_configspacen.pdf}
    \includegraphics[width=0.5\linewidth]{../../transport_eq_in_time/plots/multi_axion_canonical_configspace.pdf}
    \caption{}
\end{figure}

\begin{figure}[h]
    \includegraphics[width=\linewidth]{../../transport_eq_in_time/plots/multi_axion_powerspectrum1.pdf}
    \includegraphics[width=\linewidth]{../../transport_eq_in_time/plots/multi_axion_powerspectrum2.pdf}
    \caption{}
\end{figure}






\chapter{Results}

\section{ALP Realignment}

\begin{figure}[h]
    \includegraphics[width=\linewidth]{../../transport_eq_in_time/plots/alp_paramspace.pdf}
    \caption{}
\end{figure}

\begin{figure}[h]
    \includegraphics[width=\linewidth]{../../transport_eq_in_time/plots/alp_correct_f_a_curves.pdf}
    \caption{}
\end{figure}

\begin{figure}[h]
    \includegraphics[width=\linewidth]{../../transport_eq_in_time/plots/alp_exploration_plot0.pdf}
    \caption{}
\end{figure}

\begin{figure}[h]
    \includegraphics[width=\linewidth]{../../transport_eq_in_time/plots/alp_exploration_plot1.pdf}
    \caption{}
\end{figure}

\begin{figure}[h]
    \includegraphics[width=\linewidth]{../../transport_eq_in_time/plots/alp_exploration_plot2.pdf}
    \caption{}
\end{figure}

\section{Clockwork Theory Axions}

\begin{figure}[h]
    \includegraphics[width=\linewidth]{../../transport_eq_in_time/plots/cw_m_phi_vs_mR_1.pdf}
    \caption{}
\end{figure}

\begin{figure}[h]
    \includegraphics[width=\linewidth]{../../transport_eq_in_time/plots/cw_m_phi_vs_mR_2.pdf}
    \caption{}
\end{figure}

\begin{figure}[h]
    \includegraphics[width=\linewidth]{../../transport_eq_in_time/plots/cw_m_phi_vs_mR_3.pdf}
    \caption{}
\end{figure}

\begin{figure}[h]
    \includegraphics[width=\linewidth]{../../transport_eq_in_time/plots/cw_m_phi_vs_Gamma_inf.pdf}
    \caption{}
\end{figure}

\section{Multi Axion Systems}

\chapter{Summery and Conclusion}


\appendix
\chapter{Numerical Methods and Convergence Checks}
Text\dots
\chapter{Derivation of Background Cosmology}
Text\dots

\cleardoublepage
%% Bibliographie. Das Argument muss der Name der BIBTeX-Datenbank stehen.
%% Ein Beispiel fuer eine solche Datenbank finden Sie in bthesis_datenbank.bib
\bibliography{bthesis_datenbank}

\chapter*{Acknowledgment}
I thank Doddy Marsh and Laura Covi for the supervison and the possibility to take half a year off for a health break. Also I thank my doctors and friends. You saved my life. Finally I thank .... for finding a lot of typos, grammar and spelling mistakes.

\end{otherlanguage}
%% Dieser Befehl MUSS am Ende stehen und erzeugt die Erklaerung ueber die
%% benutzten Mittel
\Declaration
\end{document}
