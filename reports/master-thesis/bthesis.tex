\NeedsTeXFormat{LaTeX2e}[2005/12/01]
%%    2009/03/12 v1.0 GAUBM Vorlage f�r Aschlussarbeiten Physik
%% Template fuer Bachelor- und Masterarbeiten
%% an der Fakultaet fuer Physik (c) Thomas Pruschke der GA Universit�t
%% Verbesserungsvorschlaege bitte an studiendekanat@physik.uni-goettingen.de
%%
%% Benoetigte Pakete: datenumber
%%

%%%%%%%%%%%%%%%%%%%%%%%%%%%%%%%%%%%%%%%%%%%%%%%%%%%%%%%%%%%%%%%%%%%%%%
%%%%%%%%%% Bitte vor dem Veraendern diese Datei umbenennen! %%%%%%%%%%
%%%%%%%%%%%%%%%%%%%%%%%%%%%%%%%%%%%%%%%%%%%%%%%%%%%%%%%%%%%%%%%%%%%%%%

%% scrbook - Ersatz f�r LaTeX book Klasse aus dem KOMA Script
%% Moegliche Optionen: diejenigen der Klasse scrbook ausser titlepage

%% deutsche Arbeit:
\documentclass[master,       %% Typ der Arbeit: bachelor oder master
               twoside,        %% zweiseitiges Layout
               BCOR10mm,       %% Bindekorrektur 10 mm
%               liststotoc,nomtotoc,bibtotoc, %% Aufnahme der div. Verzeichnisse
                                              %% ins Inhaltsverzeichnis
               english,ngerman, %% Alternativspr. Englisch, Dokumentspr. Deutsch
%               ngerman,english  %% Alternativspr. Deutsch, Dokumentspr. Englisch
%               final,          %% Endversion; draft fuer schnelles Kompilieren
               ]{GAUBM}

\usepackage{setspace}  %% Zur Setzung des Zeilenabstandes
\usepackage{babel}     %% Sprachen-Unterstuetzung
\usepackage{calc}      %% ermoeglicht Rechnen mit Laengen und Zaehlern
\usepackage[T1]{fontenc}       %% Unterstutzung von Umlauten etc.
\usepackage[latin1]{inputenc}  %%
%% in aktuellem Linux & MacOS X wird standardmaessig UTF8 kodiert!
%\usepackage[utf8]{inputenc}    %% Wenn latin1 nicht geht ...

\usepackage{amsmath,amssymb} %% zusaetzliche Mathe-Symbole

\usepackage{lmodern} %% type1-taugliche CM-Schrift als Variante zur
                     %% "normalen" EC-Schrift
%% Paket fuer bibtex-Datenbanken
\usepackage[comma,numbers,sort&compress]{natbib}
\bibliographystyle{plainnat}

\newcommand{\tabheadfont}[1]{\textbf{#1}} %% Tabellenkopf in Fett
\usepackage{booktabs}                      %% Befehle fuer besseres Tabellenlayout
\usepackage{longtable}                     %% umbrechbare Tabellen
\usepackage{array}                         %% zusaetzliche Spaltenoptionen

%% umfangreiche Pakete fuer Symbole wie \micro, \ohm, \degree, \celsius etc.
\usepackage{textcomp,gensymb}

%\usepackage{SIunits} %% Korrektes Setzen von Einheiten
\usepackage{units}   %% Variante fuer Einheiten

%% Hyperlinks im Dokument; muss als eines der letzten Pakete geladen werden
\usepackage[pdfstartview=FitH,      % Oeffnen mit fit width
            breaklinks=true,        % Umbrueche in Links, nur bei pdflatex default
            bookmarksopen=true,     % aufgeklappte Bookmarks
            bookmarksnumbered=true  % Kapitelnummerierung in bookmarks
            ]{hyperref}

%% Weiter benoetigte Pakete: datenumber
%% Falls dieses Paket nicht in der Installation vorhanden ist,
%% kann es von der Seite mit diesem Template heruntergeladen werden
%% und in einem LaTeX bekanntem Verzeichnis installiert werden (notfalls
%% dem Verzeichnis mit der Arbeit).
\begin{document}
%%
%%                   Ab hier muessen die Anpassungen geschehen
%%
%% Hier den eigenen Namen einsetzen
\ThesisAuthor{Anna-Jana}{Rie�}
%% Hier den Geburtsort einsetzen
\PlaceOfBirth{Hamburg}
%% Titel Arbeit. Das erste Argument ist der deutsche, das zweite der
%% englische Titel.
\ThesisTitle{Spontane Axion-Leptogenese mit Generischen Kopplungen in ALP Realignment, Clockwork Axion und Multi Axion Modellen}{Spontaneous Axion-Leptogenesis with Generic Couplings in ALP Realignment, Clockwork Axion and Multi Axion Models}
%% Erst- und Zweitgutacher/in
%% Ist der/die Betreuer/in nicht identisch mit dem/r Erstgutachter/in,
%% muss diese/r als optionales Argument angegeben werden.
\FirstReferee{Prof.\ Dr.\ Laura Covi}
\Institute{Institut f\"ur Theoretische Physik}
\SecondReferee{Prof.\ Dr.\ David J. E. Marsh}
%% Beginn und Ende des Anfertigungszeitraumes
\ThesisBegin{1}{2}{2021}
\ThesisEnd{15}{8}{2022}
%% DO NOT TOUCH THESE LINES!!!!
\frontmatter
\maketitle
\cleardoublepage
%% So laesst sich in die andere Sprache umschalten (Englisch bzw. Deutsch)
\begin{otherlanguage}{english}
\begin{abstract}
  Here the key results of the thesis may be presented in about
  half a page.
  \bigskip\par
  \noindent \textbf{Keywords:} Physics, Masters Thesis, Cosmology, The Early Universe, Reheating, Axions, Baryogenesis, Spontaneous Baryogenesis, Leptogenesis, Realignment, Clockwork Theory, String Theory Axions, Non-Linear Dynamics, Thermal Field Theory, Linear Response, Transport Equation, Numerical Simulation
\end{abstract}

%% Ende des Vorspanns
\cleardoublepage
%% Ab hier 1 1/2 facher Zeilenabstand (durch setspace-Paket)
\onehalfspacing
%% Erzeugt Inhaltsverzeichnis
\tableofcontents

%%% Hier kann man seine Bezeichnungsweisen erklaeren. Falls nicht
%%% benoetigt, bis einschliesslich \end{nomenclature} auskommentieren
%\begin{nomenclature}
%%% Fuer die Berechnung der Spaltenbreiten muss \usepackage{calc}
%%% geladen sein!
%\section*{Lateinische Buchstaben}
%\noindent
%\begin{longtable}[l]{p{0.2\textwidth}p{0.7\textwidth-6\tabcolsep}p{0.1\textwidth}}
%  \tabheadfont{Variable}&\tabheadfont{Bedeutung}&\tabheadfont{Einheit}\\\midrule\endhead
%  $A$ & Querschnittsfl"ache & $\unit{m^2}$\\
%  $c$ & Geschwindigkeit & $\unitfrac{m}{s}$
%\end{longtable}
%\section*{Griechische Buchstaben}
%\begin{longtable}[l]{p{0.2\textwidth}p{0.7\textwidth-6\tabcolsep}p{0.1\textwidth}}
%  \tabheadfont{Variable}&\tabheadfont{Bedeutung}&\tabheadfont{Einheit}\\\midrule\endhead
%  $\alpha$  & Winkel & $\unit{\degree}$; --\\
%  $\varrho$ & Dichte & $\unitfrac{kg}{m^3}$
%\end{longtable}
%\section*{Indizes}
%\begin{longtable}[l]{p{0.2\textwidth}p{0.8\textwidth-4\tabcolsep}}
%  \tabheadfont{Index}&\tabheadfont{Bedeutung}\\\midrule\endhead
%  m & Meridian\\
%  $r$ & Radial
%\end{longtable}
%\section*{Abk"urzungen}
%\begin{longtable}[l]{p{0.2\textwidth}p{0.8\textwidth-4\tabcolsep}}
%  \tabheadfont{Abk"urzung}&\tabheadfont{Bedeutung}\\\midrule\endhead
%  2D & zweidimensional\\
%  3D & dreidimensional\\
%  max & maximal
%\end{longtable}
%\end{nomenclature}
%% \listoftables und \listoffigures sollten nur bei genuegender Anzahl Tabellen
%% verwendet werden
%\listoffigures
%\listoftables

\mainmatter   %% Anfang Hauptteil

\chapter{Introduction}

\chapter{Fundamentals}

\section{Early Universe Cosmology}

\begin{itemize}
	\item FLRW metric
	\item Friedmann equations
	\item power law solutions
	\item equilibrium thermodynamics with $g_*$
	\item boltzmann equations
	\item scalar fields
	\item inflation and reheating
\end{itemize}

\section{Baryogenesis}
\begin{itemize}
	\item asymmetry parameter
	\item dings da conditions
	\item B - L in the SM
	\item quik overview of mechanism 
\end{itemize}

\section{Axions}
\begin{itemize}
	\item $\theta$ parameter
	\item PQ symmetry
	\item goldstone boson 
\end{itemize}

\chapter{Spontaneous Baryogenesis}

\section{Mechanism of Spontaneous Baryogenesis}

\begin{equation}
	\mu \to \mu_\mathrm{eff} = \mu + \frac{\dot{\phi}}{f}
\end{equation}

\section{Transport Equation}

Current (continuity) equation for the charge $i$:
\begin{equation}
	\partial_\mu J_i^\mu = \sum_\alpha n_i^\alpha O_\alpha
\end{equation}

Taking the expectation value in the grand-canonical ensemble:
\begin{equation}
	\partial_t \langle J_i^0 \rangle_{\mathrm{GC}} = \sum_\alpha n_i^\alpha \langle O_\alpha \rangle_{\mathrm{GC}}
\end{equation}

which has the density matrix operator:
\begin{equation}
	\rho_\mathrm{GC} = \frac{e^{- (H - \sum_i \mu_i Q_i) / T}}{\operatorname{Tr} \left[ e^{- (H - \sum_i \mu_i Q_i) / T} \right] }
\end{equation}

\begin{align}
	\mathcal{H}(t, \mathbf{x}) &= \varepsilon \int_{- \infty}^t \, \mathrm{d} t' e^{\varepsilon (t' - t)} \sum_i \mu_i J_i^0(t', \mathbf{x}) \nonumber \\
	&= \sum_i \mu_i J_i^0(t, \mathbf{x}) - \int_{- \infty}^t \, \mathbf{d} t' e^{\varepsilon (t' - t)} \sum_i \mu_i \sum_\alpha n_i^\alpha O_\alpha(t' , \mathbf{x})
\end{align}

\begin{align}
	\frac{\mathrm{d}}{\mathrm{d} t} \mathcal{H} = \varepsilon \sum_i \mu_i J_i^0(t, \mathbf{x}) - \varepsilon^2 \int_{- \infty}^t \, \mathrm{d}t' e^{\varepsilon (t' - t)} \sum_i \mu_i J_i^0(t', \mathbf{x}) \to 0 
\end{align}

\begin{align}
	\rho \sim e^{- (H - \int \mathrm{d}^3 x \mathcal{H}) / T} = e^{-(H - \sum_i \mu_i Q_i) / T + X}
\end{align}

\begin{equation}
	X = - \sum_{i, \alpha} \frac{\mu_i}{T} n_i^\alpha \int_{- \infty}^t \, \mathrm{d}^4 x' e^{\varepsilon(t' - t)} O_\alpha(x')
\end{equation}

Pertubative expansion of $\rho$ in a series of $X$:
\begin{equation}
	\rho \approx \left[ 1 + T \int_0^{1/T} \, \mathrm{d} \tau e^{-(H - \sum_i \mu_i Q_i) \tau} X e^{(H - \sum_i \mu_i Q_i) \tau} - \langle X \rangle_\mathrm{GC}  \right] \rho_\mathrm{GC} 
\end{equation}

\begin{equation}
	\langle Q_\alpha \rangle = 0
\end{equation}

\begin{equation}
	\langle O_\alpha \rangle_\mathrm{GC} = \sum_{j, \beta} n_j^\beta \frac{\mu_j}{T} \int_{- \infty}^t \, \mathrm{d}^4 x' e^{\varepsilon (t' - t)} T \int_{0}^{1/T} \, \mathrm{d} \tau I
\end{equation}

\begin{align}
	I &= \left\langle O_\alpha(x) \left( e^{-(H - \sum_i \mu_i Q_i) \tau} O_\beta(x') e^{(H - \sum_i \mu_i Q_i) \tau} - \langle O_\beta(x') \rangle_\mathrm{GC} \right)  \right\rangle_{\mathrm{GC}} \nonumber \\
	&= \left\langle O_\alpha(x) \left( O_\beta(t' + i \tau, \mathbf{x}') - \langle O_\beta(x') \rangle_\mathrm{C} \right) \right\rangle_\mathrm{C} \nonumber \\
	&= \int_{- \infty}^{t'} \, \mathrm{d} t'' \left\langle O_\alpha(x) \frac{\mathrm{d}}{\mathrm{d} t''} O_\beta(t'' + i \tau, \mathbf{x}') \right\rangle_{\mathrm{C}}
\end{align}

\begin{equation}
	\langle O_\alpha \rangle_\mathrm{GC} = \sum_{j, \beta} n_i^\beta \mu_j \int_{- \infty}^t \, \mathrm{d}^4 x' e^{\varepsilon(t' - t)} \int_{- \infty}^{t'} \, \mathrm{d} t'' i \langle [ O_\alpha(x), O_\beta(t'', \mathbf{x}')] \rangle_\mathrm{C}  
\end{equation}

Kubo-Martin-Schwinger Relation:

\begin{align}
	\langle O_\alpha(x) O_\beta(t'' + i/T, \mathbf{x}') \rangle &= \operatorname{Tr} [ \rho_C O_\alpha(x) e^{- H / T} O_\beta(t'', \mathbf{x}') e^{+ H / T}] \nonumber \\ &= \langle O_\beta(t'', \mathbf{x}') O_\alpha(x) \rangle_{\mathrm{C}}
\end{align}

Definition spectral function:
\begin{equation}
	G^\rho_{\alpha, \beta}(x - x') = \langle [ O_\alpha(x), O_\beta(x')] \rangle_{\mathrm{C}}
\end{equation}

Fourier transform of the real space spectral function:
\begin{equation}
	G^\rho_{\alpha, \beta}(\omega, \mathbf{p}) = \int \mathrm{d}^4 x \, e^{i \omega t - i \mathbf{p} \cdot \mathbf{x}} G^\rho_{\alpha, \beta}(x)
\end{equation}

\begin{equation}
	\langle O_\alpha \rangle_\mathrm{GC} = \sum_{j, \beta} n^\beta_j \mu_j \lim_{\varepsilon \searrow 0} \int \frac{\mathrm{d} \omega}{\omega} \frac{1}{\omega - i \varepsilon} \frac{1}{i \varepsilon} G^\rho_{\alpha, \beta}(\omega, \mathbf{0}) = \sum_{j, \beta} n^\beta_j \frac{\mu_j}{T} \frac{T G^\rho_\alpha(\omega, \mathbf{0})}{2\omega} \big|_{\omega = 0}
\end{equation}

\begin{equation}
	\dot{q}_i = - \sum_j \sum_\alpha \Gamma_\alpha n_i^\alpha n_j^\alpha \frac{\mu_j}{T}
\end{equation}

Axion source term:

\begin{equation}
	\mathcal{L}_{\mathrm{int}, \beta} = - \frac{a}{f} O_\beta
\end{equation}

\begin{equation}
	H(t) \to \tilde{H}(t) = H(t) + \frac{a(t)}{f} \int \mathrm{d}^3 x  O_\beta(t, \mathbf{x})
\end{equation}

\begin{equation}
	O_\alpha(t, \mathbf{x})|_{a/f} - O_\alpha(t_0, \mathbf{x}) = i \int_{t_0}^{t} \mathrm{d} t' [\tilde{H}(t'), O_\alpha(t', \mathbf{x})|_{a / f}]
\end{equation}

\begin{equation}
	\langle O_\alpha(t, \mathbf{x})|_{a/f} \rangle_{\mathrm{GC}} - \langle O_\alpha(t_0, \mathbf{x}) \rangle_{\mathrm{GC}} = i \int_{t_0}^{t} \mathrm{d} t' \langle [\tilde{H}(t'), O_\alpha(t', \mathbf{x})|_{a / f}] \rangle_{\mathrm{GC}}
\end{equation}


\begin{align}
	& \langle O_\alpha(t, \mathbf{x})|_{a/f} \rangle_{\mathrm{GC}} - \langle O_\alpha(t_0, \mathbf{x}) \rangle_{\mathrm{GC}} = \nonumber \\  & i \int_{t_0}^{t} \mathrm{d} t' \langle [H(t'), O_\alpha(t', \mathbf{x})|_{a / f}] \rangle_{\mathrm{GC}} + \frac{a(t')}{f} i \int_{t_0}^{t} \mathrm{d} t' \langle [O_\beta(t', \mathbf{x}'), O_\alpha(t', \mathbf{x})|_{a / f}] \rangle_{\mathrm{GC}}
\end{align}

\begin{align}
	i \int_{t_0}^{t} \mathrm{d} t' [H(t'), O_\alpha(t', \mathbf{x})|_{a / f}] &\approxeq i \int_{t_0}^{t} \mathrm{d} t' \langle [H(t'), O_\alpha(t', \mathbf{x})] \rangle_{\mathrm{GC}} + i \int_{t_0}^{t} \mathrm{d} t' \langle [H(t'), O_\alpha(t', \mathbf{x})|_{a / f}] \rangle_{\mathrm{C}} \nonumber \\ &= \langle O_\alpha(x) \rangle_{\mathrm{GC}} - \langle O_\alpha(t_0, \mathbf{x}) \rangle_{\mathrm{GC}}	
\end{align}

\begin{align}
	\langle O_\alpha(x)|_{a/f} \rangle_{\mathrm{GC}} - \langle O_\alpha(t, \mathbf{x}) \rangle_{\mathrm{GC}} &\approxeq - i \int_{-  \infty}^t \mathrm{d}^4 x' \langle [ O_\alpha(x), O_\beta(x') ] \rangle_{\mathrm{C}} \frac{a(t')}{f} \nonumber \\ &= - i \int_{-  \infty}^t \mathrm{d}^4 x' G^\rho_{\alpha \beta}(x - x') \frac{a(t')}{f}
\end{align}

Axion moves slowly compared to reaction rates in the thermal plasma: 
\begin{equation}
	a(t') \approx a(t) + \dot{a}(t) (t' - t)
\end{equation}

Constant part:
\begin{equation}
	- i \frac{a_0}{f} \int_{-  \infty}^t \mathrm{d}^4 x' G^\rho_{\alpha \beta}(x - x') = - i \frac{a_0}{f} \operatorname{Pv} \int \frac{\mathrm{d} \omega}{2\pi} \frac{1}{\omega} G^\rho_{\alpha \beta} (\omega, \mathbf{0})
\end{equation}

Derivative part:
\begin{align}
	- i \frac{\dot{a}}{f} \int_{-  \infty}^t \mathrm{d}^4 x' G^\rho_{\alpha \beta}(x - x') &= - i \frac{\dot{a}}{f} \int_{-  \infty}^t \mathrm{d} t' \int \frac{\mathrm{d} \omega}{2\pi} e^{- i \omega (t - t')} \partial_\omega G^\rho_{\alpha \beta}(\omega, \mathbf{0}) \nonumber \\
	&= - i \frac{\dot{a}}{f} \lim_{\varepsilon \searrow 0} \int \frac{\mathrm{d} \omega}{2\pi i} \frac{1}{\omega - i \varepsilon} \partial_\omega G^\rho_{\alpha \beta}(\omega, \mathbf{0}) \nonumber \\
	&= \frac{\dot{a} / f}{T} \Gamma_\alpha \delta_{\alpha \beta}
\end{align}


% \section{Properties and Analytical Results}

\chapter{Motion of the Axion Field}

Action of (multiple) scalar field(s):

\begin{equation}
	S[\phi] = \int \mathrm{d}^4 x \sqrt{-g} \left[- \frac{1}{2} (\partial \phi)^2 - V(\phi) \right]
\end{equation}


Lagrange equation:

\begin{equation}
	\Box \phi \equiv \frac{1}{\sqrt{-g}} \partial_\mu \sqrt{-g} g^{\mu \nu} \partial_\nu \phi = \frac{\partial V(\phi)}{\partial \phi}
\end{equation}

Determinant of the metric:
\begin{equation}
	g = \det g_{\mu \nu} = -1 \cdot \frac{a(t)^2}{1 - kr^2} \cdot a(t)^2 r^2 \cdot a(t)^2 r^2 \sin^2 \theta = - a(t)^6 f(\mathbf{x})
\end{equation}

Computing the box operator for the metric:
\begin{align}
	    \Box \phi 
	    &= - \frac{1}{\sqrt{a(t)^6 f(\vec{x})}} (\partial_t \sqrt{ a(t)^6 f(\vec{x}) }) \dot{\phi} - \ddot{\phi}
	    = - \frac{\partial_t a(t)^3}{a(t)^3} \dot{\phi} - \ddot{\phi} \nonumber \\
	    &= - \frac{\dot{a}(t) \cdot 3 a(t)^2}{a(t)^3} \dot{\phi} - \ddot{\phi}
	    = - 3 \frac{\dot{a}}{a} \dot{\phi} - \ddot{\phi}
	    = - 3H \dot{\phi} - \ddot{\phi}.
\end{align}

Klein Gordon equation on FLRW background:
\begin{equation}
	\ddot{\phi} + 3 H \dot{\phi} - \frac{\nabla^2 \phi}{a(t)^2} + V'(\phi) = 0.
\end{equation}



\section{ALP Realignment}

generic scalar field with mass $m_a$:
\begin{equation}
	\mathcal{L} = \frac{1}{2} \left( \partial_\mu \phi \right)^2 - \frac{1}{2} m_a^2 \phi^2
	\label{eq:L_alp}
\end{equation}

The resulting eom:
\begin{equation}
	\ddot{\phi} + 3 H \dot{\phi} + m_a^2 \phi = 0 
\end{equation}

\begin{equation}
	H(t) \ll m_a
\end{equation}

\begin{equation}
	???
\end{equation}

\begin{equation}
	3 H \sim m_a
\end{equation}

�


\begin{figure}[h]
    \includegraphics[width=\linewidth]{../../transport_eq_in_time/plots/alp_realignment_field_evolution.pdf}
    \caption{Evolution of the axion field for standard misalignment of an ALP with the Lagrangian in eq. \eqref{eq:L_alp}. On the x-axis time is shown in units of the axion mass $m_a$ on a log-scale. On the y-axis the misalignment angle $\theta$ is shown. Different epochs of the evolution are annotated.}
\end{figure}

\section{Clockwork Theory Axions and Deformed Potentials}

\begin{equation}
	\mathcal{L} = \frac{f^2}{2} Z(\theta) (\partial_\mu \theta)^2 - \Lambda^4 U(\theta)
\end{equation}

\begin{equation}
	Z(\theta) = 
\end{equation}

\begin{equation}
	\phi(x) \equiv f \int_0^{\theta(x)} \mathrm{d} \theta \sqrt{Z(\theta)}
\end{equation}

\begin{equation}
	\ddot{\phi} + 3 H \dot{\phi} + \frac{\partial V}{\partial \phi}
\end{equation}

\begin{equation}
	V(\phi) = \Lambda^4 U[\theta(\phi)]
\end{equation}

\begin{equation}
	\frac{\partial V}{\partial \phi} = 
\end{equation}


\begin{figure}[h]
    \includegraphics[width=\linewidth]{../../transport_eq_in_time/plots/cw_potential.pdf}
    \caption{Plot of the deformed potentials of the clockwork axion model. On the x-axis the value of the canonical normalized field $\phi$ over the effective axion decay constant $f_\mathrm{eff}$ in the angular range $[-\pi, +\pi]$ is shown on a linear-scale. On the y-axis the value of the potential over the energy scale of the confining sector $2 \Lambda^4$ is show. The potential is shown for different values of the size of the extra dimension $mR$ indicated by the color of the plot lines. The corresponding value of $mR$ is shown on a colorbar on the right hand side.}
\end{figure}

\begin{figure}[h]
    \includegraphics[width=\linewidth]{../../transport_eq_in_time/plots/cw_field_evolution.pdf}
    \caption{Plot of the evolution of the clockwork axion field as a function of time on the x-axis in units for the inverse mass scale $1/M$ on a log-scale. The canonical field is shown on the y-axis in units of the decay constant $f$. The different epochs of the field evolution described in the main text are annotated and the final epoch of oscillations are shown as an inset plot.}
\end{figure}


\section{Multi Axion Systems}

\begin{equation}
	\mathcal{L} = \frac{1}{2} \sum_i \left( \partial_\mu \phi_i \right)^2 - \sum_i \Lambda_i^4 \left[ 1 - \cos \left( \sum_j Q_{i,j} \, \phi_j / f_j \right) \right] 
\end{equation}

\begin{equation}
	\mathcal{L} = \frac{1}{2} \sum_i \left( \partial_\mu \phi_i \right)^2 - \frac{1}{2} \sum_i m_i^2 \phi_i^2 - \sum_{i, j}
	g_{i, j} \phi_i^2 \phi_j^2
\end{equation}

\begin{figure}[h]
    \includegraphics[width=\linewidth]{../../transport_eq_in_time/plots/multi_axion_canonical_field_evolution.pdf}
    \caption{}
\end{figure}

\begin{figure}[h]
    \includegraphics[width=\linewidth]{../../transport_eq_in_time/plots/multi_axion_mass_eigenstate_field_evolution.pdf}
    \caption{}
\end{figure}

\begin{figure}[h]
    \includegraphics[width=0.5\linewidth]{../../transport_eq_in_time/plots/multi_mass_eigenstate_configspacen.pdf}
    \includegraphics[width=0.5\linewidth]{../../transport_eq_in_time/plots/multi_axion_canonical_configspace.pdf}
    \caption{}
\end{figure}

\begin{figure}[h]
    \includegraphics[width=\linewidth]{../../transport_eq_in_time/plots/multi_axion_powerspectrum1.pdf}
    \includegraphics[width=\linewidth]{../../transport_eq_in_time/plots/multi_axion_powerspectrum2.pdf}
    \caption{}
\end{figure}






\chapter{Results}

\section{ALP Realignment}

\begin{figure}[h]
    \includegraphics[width=\linewidth]{../../transport_eq_in_time/plots/alp_paramspace.pdf}
    \caption{Contour plots of the obtained baryon asymmetry  for the ALP model (shown as the decimal logarithm relative to the observed asymmetry) for the different couplings ($a\tilde{F}F)$ on the left, $a\partial J$ in the middle and $a \tilde{G} G$ on the right as a function of the axion mass $m_a$ in GeV on the x-axis on a log-scale and of the inflaton decay rate in GeV on the y-axis on a log-scale (shown on the left-most plot on the left-hand-side). The reheating temperature $T_\mathrm{RH}$ corresponding to the inflaton decay rate is shown on the right most plot on the right hand side in GeV also on a log-scale. A red line indicates the points where the observed asymmetry is reached. The values of the axion decay constant $f_a$ and Hubble scale a the end of inflation $H_\mathrm{inf}$ are given in the title. All contour plots share the same colorbar, shown below the plots.}
\end{figure}

\begin{figure}[h]
    \includegraphics[width=\linewidth]{../../transport_eq_in_time/plots/alp_correct_f_a_curves.pdf}
    \caption{The curves where the observed baryon asymmetry is reached for the ALP model for different values of the axion decay constant $f_a$ for the different couplings ($a \tilde{F} F$ on the left, $a \partial J$ in the middle and $a \tilde{G} G$ on the right). The curves are shown in the inflaton decay rate $\Gamma_\mathrm{inf}$ (shown on the y-axis on a log-scale in GeV on the left-most plot) and axion mass $m_a$ (shown on the x-axis on a log-scale in GeV). Also the reheating temperature $T_{\mathrm{RH}}$ in GeV on a log-scale corresponding to the inflaton decay rate is shown on the right-most plot.
    Each curve is colored corresponding to the value of the axion decay rate used on a log-scale. The colorbar showing the corresponding $f_a$ value is displayed on the bottom.}
\end{figure}

\begin{figure}[h]
	\includegraphics[width=0.6\linewidth]{../../transport_eq_in_time/plots/alp_minimal_possible_m_a_and_Gamma_inf_vals.pdf}
	\caption{The minimal values of the axion mass $m_a$ (solid lines) and the inflaton decay rate  $\Gamma_\mathrm{inf}$ (dashed lines) in GeV where the observed baryon asymmetry is reached as a function of the axion decay constant $f_a$ in GeV which is shown on the x-axis in units of GeV and on a log-scale. The different couplings are shown in different colors ($a F \tilde{F}$ in blue, $a \partial J$ in orange, $a G \tilde{G}$ in green).}
\end{figure}

\begin{figure}[h]
    \includegraphics[width=\linewidth]{../../transport_eq_in_time/plots/alp_exploration_plot0.pdf}
    \caption{The contour plot in the middle row shows the obtained baryon asymmetry for the ALP model for the $a \tilde{F} F$ coupling with the axion mass $m_a$ in GeV on the x-axis on a log-scale and the inflaton decay rate $\Gamma_\mathrm{inf}$ in GeV on the y-axis on a log-scale. The asymmetry is shown as the decimal logarithm relative to the observed asymmetry. A red line indicates where the observed asymmetry is reached. On the top and bottom rows the time evolution for a few example points in the parameter space of the contour plot are shown. The parameter values for of the examples are indicated by arrows into the contour plot. Each time evolution plot consists of a plot of the value of the chemical potential of the B-L charge on a log-scale as a function of time (black line). While on the bottom a plot of the reaction rate for the creation of B-L charges (blue line, log-scale, axis on the left) and of the source from the axion (orange line, linear-scale, axis on the right) is shown.
    	Both the top and the bottom plots use the same time-axis on a log-scale. On the bottom the time is shown in units of the inverse inflaton decay rate $1/\Gamma_\mathrm{inf}$ and on the top in units of the inverse axion mass $1/m_a$.}
    \label{fig:alp_explorations_plot_aFF}
\end{figure}

\begin{figure}[h]
    \includegraphics[width=\linewidth]{../../transport_eq_in_time/plots/alp_exploration_plot1.pdf}
    \caption{Same as fig. \ref{fig:alp_explorations_plot_aFF} but for the $a \partial J$ coupling.}
\end{figure}

\begin{figure}[h]
    \includegraphics[width=\linewidth]{../../transport_eq_in_time/plots/alp_exploration_plot2.pdf}
    \caption{Same as fig. \ref{fig:alp_explorations_plot_aFF} but for the $a \tilde{G} G$ coupling.}
\end{figure}

\section{Clockwork Theory Axions}

\begin{figure}[h]
    \includegraphics[width=\linewidth]{../../transport_eq_in_time/plots/cw_m_phi_vs_mR_1.pdf}
    \caption{Contour plots of the $mR$ vs $m_\phi$ parameter space of the clockwork model for the different values of the inflaton decay rate $\Gamma_\mathrm{inf}$ (in the different rows, value denoted in the y-label on the left-most plot of each row) and the axion decay rate $f_a$ (in the different columns, value denoted in the x-label of the bottom plot if each column). Each contour plot has $mR$ on the y-axis on a linear scale shown on the left most plot of each row and $m_\phi$ in eV on the x-axis on a log-scale shown on the bottom plot of each column. In each plot the baryon asymmetry as the decimal logarithm relative to the observed baryon asymmetry is shown as green-blueish colored contour lines. The axion relic density as the decimal logarithm relative to the observed dark matter density is shown as red-yellowish  colored contour lines. The curves where the observed baryon asymmetry is reached is indicated by a solid black line. The curves where the dark matter density is reached is indicated by a dashed black line. The intersection of both lines is shown as a red dot. Below the plots the colorbars for the baryon asymmetry and the relic density contour lines are shown. In each plot a red colored region indicates the part of the parameter space where the axion would start oscillating before the end of inflation invalidating our model. A orange region shows the part where the clockwork axion would decay within the life-time of the universe. }
    \label{fig:clockwork_parameter_space_aFF}
\end{figure}

\begin{figure}[h]
    \includegraphics[width=\linewidth]{../../transport_eq_in_time/plots/cw_m_phi_vs_mR_2.pdf}
    \caption{Same as fig. \ref{fig:clockwork_parameter_space_aFF} but for the $a \partial J$ coupling.}
\end{figure}

\begin{figure}[h]
    \includegraphics[width=\linewidth]{../../transport_eq_in_time/plots/cw_m_phi_vs_mR_3.pdf}
    \caption{Same as fig. \ref{fig:clockwork_parameter_space_aFF} but for the $a G \tilde{G}$ coupling.}
\end{figure}

\begin{figure}[h]
    \includegraphics[width=\linewidth]{../../transport_eq_in_time/plots/cw_m_phi_vs_Gamma_inf.pdf}
    \caption{The clockwork axion parameter space contour plots of the baryon asymmetry as the decimal logarithm relative to the observed baryon asymmetry as a function the inflaton decay rate $Gamma_\mathrm{inf}$ in GeV on the y-axis on a log-scale and the axion mass $m_\phi$ in eV on the x-axis on a log-scale.
    Each row has a different coupling namely $a F \tilde{F}$ in the top row, $a \partial J$ in the middle row and $a G \tilde{G}$ in the bottom row. Each column has a different value of the size of the extra dimension $mR$ indicated on the x-label of the bottom plot of each column. Each plot has its own colorbar for the contour lines as the scales are very different for each plot. A red line is used to indicate the curves where the observed baryon asymmetry is reached.}
\end{figure}

\begin{figure}[h]
	\includegraphics[width=\linewidth]{../../transport_eq_in_time/plots/cw_exploration_plot.pdf}
	\caption{A contour plot of the clockwork axion mass $m_\phi$ vs size of the extra dimension $mR$ parameter-space is shown in the middle row. The baryon asymmetry is shown in green-blueish contour lines. The relic density is shown in red-yellowish contour lines. Both are shown are the decimal logarithm relative to the observed value. The red colored area indicates the part of the parameter-space where axion oscillates before the end of inflation and hence invalidates the model. The orange area indicates the that the axion decays within the life-time of the universe. In the top and bottom rows the time evolution of the processes corresponding to example points in the parameter space indicated by arrows is shown. Each of these plots consists of three subplots. On the left the evolution of the canonical clockwork field until the oscillations is shown. On the right the time evolution of the chemical potential over the temperature and the rate of B-L generation and the axion source as in fig. \ref{fig:alp_explorations_plot_aFF} are shown.}
\end{figure}

\section{Multi Axion Systems}

\chapter{Summery and Conclusion}


\appendix

\chapter{Some Appendix ....}

\cleardoublepage
%% Bibliographie. Das Argument muss der Name der BIBTeX-Datenbank stehen.
%% Ein Beispiel fuer eine solche Datenbank finden Sie in bthesis_datenbank.bib
\bibliography{bthesis_datenbank}

\chapter*{Acknowledgment}
I thank Doddy Marsh and Laura Covi for the supervison and the possibility to take half a year off for a health break. Also I thank my doctors and friends. You saved my life. Finally I thank .... for finding a lot of typos, grammar and spelling mistakes.

\end{otherlanguage}
%% Dieser Befehl MUSS am Ende stehen und erzeugt die Erklaerung ueber die
%% benutzten Mittel
\Declaration
\end{document}
